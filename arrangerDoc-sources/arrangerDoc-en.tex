%%%%%% English Documentation of arranger.ly. Version YYYY/MM/DD = 2022/12/16 %%%%%%
%% To compile with : ConTeXt (LuaTex)
%% https://wiki.contextgarden.net/Main_Page

%\setupsynctex[state=start,method=min] : Don't work anymore !!!
\setuplayout[header=1cm, topspace=0.5cm, margin=0.5cm,  height=middle, width=middle]
%\showframe
%\showlayout
%\showsetups

\language[en] % english
\definebodyfontenvironment[default][em=italic]
\setupbodyfont[11pt]

\setuppagenumbering[location={footer}, left={-},right={-}]
\setuphead[title][align=center,style=bold, 
                  incrementnumber=yes,  % keep track of the number
                  number=no]
\setuphead[subject][textstyle={\bold},after=\nowhitespace, %
                            incrementnumber=yes,  % keep track of the number
                            number=no]
%\setuphead[section][after=\nowhitespace,before=\nowhitespace]  
\setuphead[subsection][before={\blank[-3.5em]}] %  \nowhitespace is not enough
                          
\setupitemize[before=\nowhitespace]

\setuplist[title][align=center,style=bold, alternative=a, pagenumber=no, before={\blank[0.3cm]}, after={\blank[0.1cm]}]
\setuplist[subject,chapter,subject,subsubject,section,subsection][alternative=c]
\setuplist[subsection][margin=1.5em]
%\setuplist[title][align=center,style=bold, alternative=a]
%\setupheadtext[content=Sommaire]
\setupcombinedlist[content][list={title,chapter,subject,subsubject,section,subsection}]


\def\lilyspace{0.7cm}
\definehspace[lilyindent][0.7cm]
\definehspace[threelilyindent][2.1cm]
\definetyping[lily][margin=\lilyspace, page=yes, escape={&,&}]
\definetyping[mylily][margin=\lilyspace, space=fixed, before=\nowhitespace, after=\nowhitespace, escape={&,&}]

\def\myindent{\hphantom{0}\hspace[lilyindent]}

\defineframed[noframed][frame=no]

%% URL
\useURL[chord][http://gillesth.free.fr/Lilypond/chordsAndVoices/]
\useURL[changePitch][http://gillesth.free.fr/Lilypond/changePitch/]
\useURL[copyArticulations][http://gillesth.free.fr/Lilypond/copyArticulations/]
\useURL[addAt][http://gillesth.free.fr/Lilypond/addAt/]
\useURL[extractMusic][http://gillesth.free.fr/Lilypond/extractMusic/]
\useURL[checkPitch][http://gillesth.free.fr/Lilypond/checkPitch/]
\useURL[copyArticulations-LSR][http://lsr.di.unimi.it/LSR/Item?id=769]
\setupinteraction[state=start,color=black,contrastcolor=black, style=\tf]

\setupexternalfigures[directory={exemples}]   % images (sub)directory

%%%%% function \syntax
\define[2]\syntax
{% display => syntax : #1 #2
\blank
\hbox{\hspace[lilyindent]{\em {\small $\triangleright$} syntax }{: }% 
\doifemptyelse{#2}%
{\inframed{\type{#1}}}%
{\vcenter\framed[location=middle, width=fit, height=fit]{%% \inframed doesn't center vertically :-(
\vtop{{\hbox{\type{#1}}}%
      {\hbox {\hspace[threelilyindent] {\ttsl \#:optional} \type{#2}}}}}}%
}%
\blank
}

%%%%% function \func
\define[1]\func
{ % display => The function #1
 %% \index{#1}   % <- #1 can end with character + which will not be displayed in index ! :-(
 %% (bad) workaround :
 \ctxlua{context.index((string.gsub("#1","+", "†")))}  %%  replace in index, any + character by †
                           %% Note the double parenthesis to force a single return value  of gsub
                           %% https://wiki.contextgarden.net/Programming_in_LuaTeX
 \startsubsection[number=no,list=#1,reference=#1]{}
 %\startsubsubject[list=#1,reference=#1]{}   % don't work !
 {\switchtobodyfont[20pt] \checkmark} 
 {\sca the function \cap{\bf #1}}
 %\stopsubsubject
 \stopsubsection
}
%%%%%%%%%%%%%%%%%%%%%%%%%%%%%%%%%%%%% Start of document %%%%%%%%%%%%%%%%%%%%%%%%%%%%%%%%%%%%%%%%%
\starttext

%%%%%% Title page
\startstandardmakeup
\midaligned{\tfd\sc Arranger.ly}
\blank[2cm]
\midaligned{\tfd ---}
\stopstandardmakeup

%%%%% Contents
\midaligned{\bf \underbar Contents}
\placecontent[criterium=all]

%%%%%% Chapter 1 ...
\title[generalites]{OVERVIEW}
\subject[objectifs]{Basic goals}
{\em arranger.ly provides an environment facilitating musical arrangement.} \footnote{To arrange
herein means to re-orchestrate an original instrumentation.} A set of functions enables quick
re-orchestration of a piece of music, using a minimal and reusable music encoding.
\blank[0.7em]
One of the main aspects of {\em arranger.ly} concerns the locating system of musical positions, which is  now based on {\em bar numbers}\footnote {Lilypond use a system based on {\em moment}s : \type{(ly:make-moment 5/4)} for example.}.The arranger's workflow is made more flexible : rather than entering music expressions instrument by instrument in a linear fashion, it becomes possible to work as the ideas go by -- first deal with the melody, then accompaniment,
then the bass, etc.
\blank[0.7em]
The user typically first declares a list of instruments. {\em arranger.ly} takes care of initializing
each instrument with empty measures. Then, in a single command, the user can insert a music fragment
in several instruments and positions, as well as \quotation{copy-paste} entire music sections in one
line of code.

Functions allow for octave transposing and octave doubling, specifying patterns for repeated rhythms or articulations, distributing the notes to various instruments in
a succession of chords, inverting chords,~\dots, so as never to repeat information.

All these functions can be directly used from Scheme, which makes for lighter syntax (no backslash before
variable names) and easier editing of instrument lists.

Once the arrangement is finished, it can be exported to usual LilyPond source:
\blank[0.3em]
\startmylily
flute = {…}
clar = {…}
…
\stopmylily
\godown[-0.3em]

%%%%%%%%%%%%%%%%%%%%%%%%%%%%%%%%%%%%%%%%%%%
\subject{Software dependencies}
\startitemize[2,joinedup,intro]
\item You need LilyPond 2.20 or higher.
\item The file {\em arranger.ly} requires the following \type{include} files:
  \starttabulate[|l|l|,before={\blank[1mm]},after={\blank[1mm]}, indenting=yes]
    \NC {\textbullet \hspace[1] \em chordsAndVoices.ly} \NC\small {(\from[chord])} \NC
    \NR
    \NC {\textbullet \hspace[1] \em changePitch.ly} \NC\small{ (\from[changePitch])} \NC
    \NR
    \NC {\textbullet \hspace[1] \em copyArticulations.ly} \NC\small{(\from[copyArticulations])} \NC
    \NR
    \NC {\textbullet \hspace[1] \em addAt.ly} \NC\small{(\from[addAt])} \NC
    \NR
    \NC {\textbullet \hspace[1] \em extractMusic.ly} \NC\small{ (\from[extractMusic])} \NC
    \NR
    \NC {\textbullet \hspace[1] \em checkPitch.ly} \NC\small{ (\from[checkPitch])} \NC
    \NR
  \stoptabulate 
\stopitemize

It is easiest to put these 6 files in the same folder alongside with {\em arranger.ly},
and call LilyPond with option \type{--include=myfolder}. Only the following line should then
be added at the top of one's \type{.ly} file:\\
\myindent\type{\include } \type{"arranger.ly"} % spaces pbs with \type{\include "arranger.ly"}
\godown[-0.3em]

%%%%%%%%%%%%%%%%%%%%%%%%%%%%%%%%%%%%%%%%%%%
\subject{Two prerequisites to using the functions}
\startitemize[n]
\item Have all meter changes in a \type{\global} variable, e.g.:
\startmylily global = { \time 4/4 s1*2 \time 5/8 s8*5*2 \time 3/4 s2.*2 } \stopmylily
This enables {\em arranger.ly} to convert all measure numbers to LilyPond moments.
\item Use the \goto{\type{init}}[init] command described at \at{page}[init] to declare
instrument names to the parser. This needs to be placed before any call to the functions
described below.
\stopitemize
\page 
%%%%%%%%%%%%%%%%%

\subject{Conventions and reminders}
In this document, we shall call {\em instrument} any Scheme symbol referencing
a LilyPond music expression. The music an instrument points to has the same length
as \type{\global} and begins at the same time (by default, this is measure 1, with an
optional upbeat). However, in the following text, {\em music} more generally refers to
a fragment with indeterminate position, which can be inserted at any measure in the piece.

Being a symbol, an instrument is denoted in Scheme using a leading \hbox{single vertical
quote \type{'}}\\
\myindent ex : \type{'flute}\\
In running LilyPond input, it additionally needs to be prefixed with a hash sign \type{#} in order to be recognized as a Scheme expression.\\
\myindent ex : \type+#'flute+\\
The bare name \type{flute} in Scheme is equivalent to \type{\flute} in LilyPond.

In Scheme code, a list of instruments can be written as either

\startmylily '(flute oboe clarinet)\stopmylily

or

\startmylily(list 'flute 'oboe 'clarinet)\stopmylily

A list of music expressions is written as

\startmylily(list flute oboe clarinet)\stopmylily

or using a so-called \quotation{quasiquote}:

\startmylily 
`(,flute ,hautbois ,clarinette) ;&\em{& note the use of `( instead of '( &}&
\stopmylily
These lists can be manipulated with ease thanks to {\em arranger.ly}'s utility functions
(see \type{lst}, \type{flat-lst} and \type{zip}).
\blank[-0.6cm]                 
%%%%%%%%%%%%%%%%%%

%%%%%%%%%%%%%%%%%%

\reference[init]{}
\index[init-info]{init}                   
\subject{Initialization}
- The \type{init} function must be called \emph{after} declaring \type{\global} and
\emph{before} any call to the other functions. It is passed a list of instruments and
an optional integer.

\syntax {(init instru-list}{measure1-number)}

Each instrument in the list is declared to LilyPond and filled in with multi-measure rests.

If \type{\global} was defined using:
\startlily
global = { s1*20 \time 5/8 s8*5*10 \bar "|."}
\stoplily
the following code:
\startlily
all = #'(flute clar sax tptte cor tbne basse)
#(init all)
\stoplily
is equivalent to
\startlily
flute = { R1*20 R8*5*10 } 
clar = { R1*20 R8*5*10 } 
sax = { R1*20 R8*5*10 } 
tptte = { R1*20 R8*5*10 } 
cor = { R1*20 R8*5*10 } 
tbne = { R1*20 R8*5*10 } 
basse = { R1*20 R8*5*10 }
\stoplily
\blank[0.8em]

- \type{instru-list} may be empty: \type{(init '())}. A noteworthy use case
is direct editing of the \type{\global} variable, as shown in \goto{addendum I}[addendum1]
at page \at{page}[addendum1].

Once all music events influencing the meter are declared in \type{\global}, \type{init}
can be called a second time with a non-empty instrument list.
\page
- To count measures, \type{init} takes into account manual overrides applied to properties
of the \type{Score} context and the \type{Timing} object, such as \type{measurePosition},
\type{measureLength}, \type{currentBarNumber}, as well as the \type{\partial} and
\type{\cadenzaOn|Off} commands. If \type{\partial} is placed at the very beginning of the piece,
\type{init} even adds a rest with same duration as the pickup to all the instruments.

\setupcaptions[location=none]
\startfiguretext[right][exemple1]{}
{\vbox{\blank[-0.6em]\externalfigure[exemple01.pdf][wfactor=180, align=flushright]}}
\underbar{\sc\Words example 1}
\startlily[bodyfont=10.7pt]
global = { 
  \partial 4 s4
  s1*2
  %&\em{& measure &}&3&\em{& : only &}&2&\em{& beats&}&
  s4 \set Timing.measurePosition =
                  #(ly:make-moment 3/4)
     s4
  s1                    %&\em{& measure &}&4
  \set Score.currentBarNumber = #50
  %&\em{& \set Timing.currentBarNumber = #&}&50
  s1                    %&\em{& measure &}&50&\em{& !&}&
  \bar "|."  }
all = #'(fl cl sax tptte cor tbne basse)
#(init all)

\stoplily
\stopfiguretext

\index{measure-number->moment}
The internal function \type{measure-number->moment} may be used to ensure that
{\em arranger.ly} and {\em LilyPond} stay in sync. For example,
\startmylily
#(display (map measure-number->moment '(1 2 3 4 50)))
\stopmylily
prints the number of quarter notes elapsed from music start for measures 1, 2, 3, 4 and 50:
\startmylily
(#<Mom 1/4> #<Mom 5/4> #<Mom 9/4> #<Mom 11/4> #<Mom 15/4>)
\stopmylily
\blank[1em]  % crlf 
- The optional parameter \type{measure1-number}\crlf
\type{init} accepts an integer as optional last argument, indicating  the
numbering of the first measure. It defaults to 1. This is useful to add, say
3 measures of intro to the arrangement.
\startmylily 
(init all -2) 
\stopmylily
This automatically shifts all previously entered measure positions. In this case,
it is relevant while arranging to add
\startmylily
\set Score.currentBarNumber = #-2
\stopmylily
at the beginning of \type{\global}, and let \type{measure1-number} default to 1.
Then, once the arrangement is finished, this line can be removed while \type{measure1-number}
is set to -2.

From a general point of view, the following settings are useful while working:
\blank[1em]
\startmylily
tempSettings = { 
  \override Score.BarNumber.break-visibility = ##(#f #t #t)
  \override Score.BarNumber.font-size = #+2
  \set Score.barNumberVisibility = #all-bar-numbers-visible 
}
\stopmylily

%%%%%%%%%%%%%%%%%%%%%%%%%%%%%%                     
\reference[rm-info]{}           
\subject{The basic function: \type{rm}}
%\label{rm}%
\type{rm} means \quotation{\underbar{r}eplace \underbar{m}usic}.
This function typically redefines an {\em instrument}, replacing part of
its existing music with the music fragment given as an argument.\crlf
\type{rm} is actually an extension of \type{\replaceMusic} from {\em extractMusic.ly}.
Optional reading is chapter~8 from this file's documentation at~:\crlf
\hbox{\hspace[lilyindent]\from[extractMusic]}
\page %%%%%%%%%%%%%%%%%%%%%%%%
Below is the syntax of \type{rm} :

\syntax{(rm obj where-pos repla}{repla-extra-pos obj-start-pos)}

- \type{obj} is $\left\{
\vcenter{%
\mbox{an {\em instrument}, e.g. \type{'flute }}
\mbox{a list of {\em instrument}s : \type{'(clar tpt sax)}}
\mbox{but may also be a {\em music} : \type{music or #{…#}} }
\mbox{or a list of {\em music}s : \type{(list musicA musicB musicC) }}
}
\right.$
\blank[medium]
- \type{where-pos} indicates the bar where replacement is performed. More precisely,
it is a  \startnarrower[left] {\em music position} as defined in the next paragraph (\at{page}[positions_musicales]).\stopnarrower

- \type{repla} is a {\em music} or a list of {\em music}s, but syntax with \type{quote '} is valid~:
\startnarrower[left] \type{'(musicA musicB musicC…)}.\stopnarrower

- \type{repla-extra-pos }and\type{ obj-start-pos } are {\em music positions} too  (read on).
\blank
{\hbox{\hspace[lilyindent]{\em {\small $\triangleright$} return }{: }}}
\blank
- If \type{obj} is an {\em instrument} or a {\em music}, \type{rm} returns the music
obtained after performing the \startnarrower[left] replacement. In the case of an {\em instrument}, this new value is automatically reassigned to the symbol representing it.\stopnarrower

- If \type{obj} is a list of {\em instrument}s or {\em music}s, \type{rm} returns the list of the obtained {\em music}s.
\blank[0.7cm]
\startfiguretext[right][exemple2]{}
{\vbox{\blank[-1.1em]\externalfigure[exemple02-en.pdf][wfactor=175, align=flushright]}}
\underbar{\sc\Words exemple 2}
\blank[0.7cm]
\startlily
global = { s1*4 \bar "|." } 
all = #'(fl cl sax tpt horn tbn bass) 
#(init all)

musA = \relative c' { e2 d c1 }
musB = { f1 e1 } 
musC = { g,1 c1 }

#(begin 
  (rm 'fl 1 #{ c'''1 #})
  (rm '(cl sax tpt) 2 #{ c''1 #}) 
  (rm '(horn tbn bass) 3 
                  '(musA musB musC)))
\stoplily %\blank[2.1cm]%
\stopfiguretext
\blank[0.7cm]
By default, the \type{rm} function accounts for the entire music given in \type{repla}.
It is however possible to take only a part of it by specifying the optional
parameter \type{repla-extra-pos}.\crlf
The principle is as follows:
\hairline
\type{repla} is positioned at the lowest position between \type{where-pos} and \type{repla-extra-pos} :
\definesymbol[5][$\rightarrow$]
\startitemize[5, joinedup, intro][margin=\lilyspace]
\item if \type{repla-extra-pos} is before \type{where-pos},
the part {[}\type{repla-extra-pos}, \type{where-pos}{[}
is {\em not} replaced. The beginning of \type{repla} is ignored.
\item If \type{where-pos} is before \type{repla-extra-pos},
only {[}\type{where-pos}, \type{repla-extra-pos}{[}
from the instrument is replaced, and the end of \type{repla} is ignored.
\stopitemize
\hairline

\blank[0.7cm]
Examples are most intuitive:
\page %%%%%%%%%%%%%%%%%%%%
\underbar{\sc\Words exemple 3}
\startfiguretext[right][exemple3]{}
{\vbox{\blank[0.15em]\externalfigure[exemple03.pdf][wfactor=220, align=flushright]}}
\startlily
mus = \relative c' {
  f1 c' f a f' }    %&\em{& cl mes &}&4  
#(begin
  (rm 'fl 7 mus 4)
  (rm 'cl 4 mus #f) 
    ;&\em{& (rm 'cl &}&4&\em{& mus)&}&
  (rm 'bs 4 mus 6))
\stoplily
\stopfiguretext

- Optional parameter \type{obj-start-pos} may precise where \type{obj} begins
(\type{repla-extra-pos}, above, related to \type{repla}). Typically here, \type{obj} is a
{\em music} rather than an {\em instrument} and the return value of \type{rm} is used.\\

In example 3, we change now the F note at bar 6 into an E flat, assigning the
result to another instrument, a sax.

\defineparagraphs[mypar][n=3,before={\blank}, after={\blank}, distance=1em]
\setupparagraphs[mypar][1][width=0.42\textwidth, style=type]
\setupparagraphs[mypar][2][width=0.07\textwidth, style=type] %, rule=on]
\setupparagraphs[mypar][3][width=0.45\textwidth, style=italic]
\startmypar
%\vbox{
\startlily                
#(let((m (rm mus 
             6 #{ ees'1 #} 
             #f
             4)))
    (rm 'saxo 4 m))
\stoplily
%}%
\mypar 
; let\crlf
; 6\crlf
; \#f\crlf
; 4
\mypar
declares local variables\crlf
bar that is replaced with an E flat\crlf
repla-extra-pos,\crlf
position where music begins
\stopmypar
\godown[-0.2em]
\hairline
Do note the difference between \type{(rm music…)} and \type{(rm 'music…)}. The former
returns a new musical expression without actually modifying \type{music}, whereas the
latter assigns this return value back to the \type{'music} instrument.
\hairline
\blank
- In case \type{obj} is a list of {\em instrument}s, any element of this list may in turn
be a list of {\em instrument}s. Thus,
\startmylily
(rm '(flute (clar sax) bassClar) 5 '(musicA musicB musicC))
\stopmylily 
will trigger assignments as in this diagram:\crlf
\vtop{\hbox{
 \hspace[lilyindent]
 \starttable[|lT|c|l|]
    \type{'flute} \NC $\leftarrow$ \NC \type{\musicA} \FR
    \NR
    \type{'clar} \NC $\leftarrow$ \NC \type{\musicB} \MR
    \NR
    \type{'sax} \NC $\leftarrow$ \NC \type{\musicB} \MR
    \NR
    \type{'bassClar  } \NC $\leftarrow$ \NC \type{\musicC} \LR
    \NR
  \stoptable}
}
\godown[-0.1em]

                     %%%%%%%%%%%%%%%%%%%%%%%%%%%%%%
\subject[positions_musicales]{Music positions and bar numbers explained}
- A position is denoted by a bar number. What if the position should not begin at
the start of a measure? In such a case, the position is a {\em list of integers}: 
\startmylily '(n i j k …) \stopmylily
where \type{n} is the bar number, and \type{i j k …} are powers of two (1, 2, 4,
8, 16, etc…) denoting the distance from the beginning of the \type{n}-th bar.\footnote{The \goto{\type{add-dynamics} function}[add-dynamics] \at{page}[add-dynamics], show some pariculars cases where \type{i j k} … are integers but not a power of two.} 

Thus, \type{'(5 2 4)} is a position, located in measure 5, after a half note (\type{2})
and a quarter note (\type{4}), that is, in a \type{4/4} beat, fifth measure, fourth beat.\\

- Any \type{n} lower than the \type{measure1-number} passed to \type{init}, which
defaults to 1, will be silently transformed into that number. In practice, it means that
\type{'(0 2 4)} points to the same location as \type{'(1 2 4)}…\\

- {\em Negative} values for \type{i j k …} are allowed. In a 4/4 beat, \type{'(5 2 4)} is the same position as \type{'(6 -4)}, which reads \quotation{One quarter note before measure 6.}.
Negative values are the only way to access a pickup at the start of the piece: \type{'(1 -4)} is the beginning of a tune starting with \type{\partial 4 …}.
\page %%%%%%%%%%%%%%%%%%%%%%%%%%%%
- Like the expressions for durations in Lilypond, a dot after a power of 2 is possible:
\type{'(7 4.)} instead of \type{'(7 4 8)}\crlf
For 2 points and more, however, we should write: \type{'(7 4.2)} for \type{'(7 4 8 16)}, \type{'(7 4.3)} for \type{'(7 4 8 16 32)}, etc... (up to 9 points!).
\blank
- Any note still held at the beginning of the replacement is appropriately shortened by \type{rm}.

In the \goto{previous example \at{(page}{)}[exemple3]}[exemple3], this code: 
\startmylily
(rm 'cl '(5 2 4) #{ r4 #})
\stopmylily
would yield, as the clarinet's fifth measure, to:
\startmylily 
{c2. r4}
\stopmylily
$\Longrightarrow$ the whole C note turns into a dotted half note.
\reference[multirestWarning]{}
\hairline
Beware: while notes and rests may be arbitrarily split into smaller values,
full-measure rests (written with capital \type{R}) can only be shortened at bar lines.
\hairline
\blank[0.5em]
This is why, in our \goto{example 3 on page \at{}[exemple3]}[exemple3],
\startmylily
(rm 'fl '(5 2 4) #{ c''4 #})
\stopmylily
would trigger a warning resembling:
\startmylily
&\it\quotation{{&warning: barcheck failed at: 3/4
 mmR = { #infinite-mmR \tag #'mmWarning R1 }&}}&
\stopmylily
(The 2\high{nd} line originates from the {\em extractMusic.ly} file.)\\
The solution is:
\startmylily
(rm 'fl 5 #{ r2 r4 c''4 #})    ;&\em{& rests written out by hand !&}&
\stopmylily
\blank

- This last example demonstrate the use of positions with the \type{\cadenzaOn} command.
\blank[0.8em]
\underbar{\sc\Words exemple 4}
\godown[-0.2em]
\startlily
cadenza = \relative c' { c4^"cadenza" d e f g }
\stoplily
\godown[-1.7em]
\startfiguretext[right][exemple4]{}
{\vbox{\blank[-0.5em]\externalfigure[exemple04.pdf][wfactor=200, align=flushright]}}
\startlily
global = {
   \time 3/4
   s2.
\stoplily
\stopfiguretext
\godown[-3em]
\startlily
   \cadenzaOn #(skip-of-length cadenza) \bar "|" \cadenzaOff
   s2.*2 \bar "|." }
   
#(begin (init '(clar))
        (rm 'clar 2 cadenza)
        (rm 'clar 3 #{ c'2. #}))
\stoplily

In order to insert an E note before measure 3, one can use negative number:
\startmylily
(rm 'clar '(3 -2 -4) #{ e'2. #})
\stopmylily

Internally, {\em arranger.ly} occasionally uses a different syntax for positions:
\startmylily `(n ,moment)    ;&\it\rm{& or : (list n moment)&}& \stopmylily

To insert the E, the following would then be possible:
\startmylily
(rm 'clar `(2 ,(ly:music-length cadenza)) #{ e'2. #})
\stopmylily
Note finally that the syntax \type{`(n ,(ly:make-moment p/q))} can be reduced to \type{'(n p/q)}, provided that the quotient \type{p/q} is not reducible to an integer.
\startmylily 
(rm 'clar '(2 5/4) #{ e'2. #})  ;&\it\rm{& ok with 5/4 : same result as previous code&}&  \stopmylily
On the other hand, \type{8/4} would be \type{(ly:make-moment 1/2)}, not \type{(ly:make-moment 2/1)}.
\blank[1.2em]
- \underbar{Convention} :
\blank[0.3em]
\hairline
In all following functions, any argument ending in \type{-pos} (such as
\type{from-pos}, \type{to-pos}, \type{where-pos}, etc.) shall be \type{position}s as
described in this paragraph, as well as \type{pos1}, \type{pos2}, etc\dots
\hairline
\page %%%%%%%%%%%%
%%%%%%%%%%%%%%%%%%%%%%%%%%%%%%%%%%%%%%%%%%%%%%%%%%%
\title[ListageFonctions]{LISTINGS of the FUNCTIONS}

                      %%%%%%%%%%%%%%%%%%%%%%%%%%%%%%                                  
\subject[copiercoller]{Copy-paste functions}
\blank[-2em]
%%%%%%%%%
\func{rm}
\syntax{(rm obj where-pos repla}{repla-extra-pos obj-start-pos)}
\type{rm} is described separately in a very detailed manner \at{page}[rm-info].
\blank[1em]
\midaligned{ \tfd ------------}
\godown[0.5em]

%%%%%%%%%
\func{copy-to}
\index{copy-to-with-func}
\syntax {(copy-to destination source from-pos to-pos . args)}{}
Copy \type{source} in \type{destination} between positions \type{from-pos} and \type{to-pos}\\
\type{destination} can be an {\em instrument}, or a list of a mix of {\em instrument}s and lists of {\em instrument}s.\\
\type{source} is an {\em instrument}, or a list of {\em instrument}s, but also a {\em music} or a list of 
{\em music}s\\
You can put after several sections, by specifying new sources and new positions in the parameter optional \type{args}. User can optionally separate each section by a slash \type{/}.
\startmylily
(copy-to destination sourceA posA1 posA2 / sourceB posB1 posB2 / etc…)
\stopmylily
If you omit the parameter \type{source} in a section, the source of the previous section 
is taken into account.
\startmylily
(copy-to destination source pos1 pos2 / pos3 pos4)
\stopmylily
is equivalent to : 
\startmylily
(copy-to destination source pos1 pos2 / source pos3 pos4)
\stopmylily
If \type{source} do not begin at the beginning of the piece, then the optional key parameter\\ \type{#:source-start-pos} can be used like that:
\startmylily
(copy-to dest source pos1 pos2 #:source-start-pos pos3 / pos4 pos5 …)
\stopmylily
Finally, user can replace \type{copy-to} by the function \type{(copy-to-with-func func)}, which will apply \type{func} to each copied section. See how to use this feature at the function  \goto{\type{apply-to}}[apply-to], \at{page}[apply-to].
\startmylily
((copy-to-with-func func) destination source pos1 pos2 …)
\stopmylily
\blank[1.2em]
\midaligned{ \tfd ------------}
\godown[0.7em]

%%%%%%%%%
\func{copy-out}
\index{copy-out-with-func}
\syntax {(copy-out obj from-pos to-pos where-pos . other-where-pos)}{}
Copy out the section \type{[from-pos to-pos[} of the instrument or list of instruments \type{obj}, to the position \type{where-pos}, and then eventually to other positions.
\startmylily
(copy-out obj from-pos to-pos where-pos1 where-pos2 where-pos3 etc...)
\stopmylily
User can replace \type{copy-out} by the function \type{(copy-out-with-func func)}, which will apply \type{func} to each copied section. See how to use this feature at the function  \goto{\type{apply-to}}[apply-to], \at{page}[apply-to].
\startmylily
((copy-out-with-func func) obj from-pos to-pos where-pos …)
\stopmylily
\blank[1.2em]
\midaligned{ \tfd ------------}
\godown[0.7em]

%%%%%%%%%
\func{x-rm}
\syntax {(x-rm obj replacement pos1 pos2 … posn)}{}

Simple shortcut for :
\startmylily
(rm obj pos1 replacement)
(rm obj pos2 replacement)
…
(rm obj posn replacement)
\stopmylily
\blank[0.5em]
\midaligned{ \tfd ------------}
\page %%%%%%%%%%%%%%%%%%%%%%%%%%%%%%%%%%%%%%%%%%%%%%%%%%%%%%%%%%%%

%%%%%%%%%
\func{rm-with} 
\syntax {(rm-with obj pos1 repla1 / pos2 repla2 / pos3 repla3 …)}{}

Shortcut for : 

\startmylily
(rm obj pos1 repla1)
(rm obj pos2 repla2)
etc…
\stopmylily
The slash \type{/} that split the instruction is optional.\\
If a \type{repla}{\em n} want to use music of a previous section, once modified, please use the scheme function \type{delay} and the function \type{em} of the \at{page}[em] in the following way~:
\startmylily
(delay (em obj pos1 …)) ; &\tfx\em{&Extract obj music after it is modified&}&
\stopmylily 
\blank[0.8em]
\midaligned{ \tfd ------------}
\godown[0.7em]

%%%%%%%%%
\func{apply-to}
\index{to-set-func}
\index{compose}
\syntax{(apply-to obj func from-pos to-pos}{obj-start-pos)}
Apply \type{func} to music of \type{obj} inside section \type{[from-pos to-pos[}.\\
\type{obj} is a {\em musique}, an {\em instrument}, or a list of {\em musique}s or {\em instrument}s.\\ 
The \type{obj-start-pos} parameter allows user to specify the starting position of \type{obj} when  different from the whole piece.
\blank[0.5em]
\underbar{The parameter \type{func}} :
\blank[0.7em]

{\tfa -} \type{func} is a function with only one parameter of type \type{music}.\\
"{\em arranger.ly}" defines a number of such function, in the form of a sub-function whose name begins with \type{set-}~: \type{set-transp}, \type{set-pat}, \type{set-ncopy}, \type{set-note}, \type{set-pitch}, \type{set-notes+}, \type{set-arti}, \type{set-reverse}, \type{set-del-events}, \type{set-chords->nmusics}.\\
(These functions are described later in this document).
\blank[0.6em]

{\tfa -} You can, however, easily create your own functions, compatible \type{apply-to}, with the help of a "wrapper" function called \type{to-set-func}, particularly adapted to changing musical properties. \type{to-set-func} takes itself in parameter, a {\em function} with musical parameter.\\
In the following example, we define a function \type{func} which, when used with \type{apply-to}, will transform all \type{c'} into \type{d'}.
\blank[0.3em]

\startmylily
(define func (to-set-func (lambda(m) 
               (if (equal? (ly:music-property m 'pitch #f) #{ c' #}) 
                 (ly:music-set-property! m 'pitch #{ d' #})))))
\stopmylily
%\crlf % don't work
\blank[0.3em]
{\tfa -} You can also group several operations together at the same time, using the \type{compose} function~:
\blank[0.3em] 
\startmylily (compose func3 func2 func1 …) \stopmylily 
\blank[0.3em]
…which will result, when applied to a \type{music} parameter, to :
\blank[0.3em]
\startmylily (func3 (func2 (func1 music)))\stopmylily
\blank[0.6em]
{\tfa -} Let's go back to the functions of "{\em arranger.ly}" mentioned earlier, functions of the form :
\blank[0.3em] 
\startmylily ((set-func args) music) \stopmylily
\blank[0.3em]
During the call of \type{apply-to}, all arguments of the sub-function \type{set-func} remain the same and fixed for all instruments contained in \type{obj}. 
However, it is in certain cases desirable that these arguments are, on the contrary, customizable for each instrument.\\
This will be possible, provided that a new syntax is adopted for the argument \type{func} of \type{apply-to}, which will then be defined as a pair, with in 1\high{st} element, the name of the sub-function, and in 2\high{nd}, a list, composed with the arguments corresponding to each instrument.

\page %%%%%%%%%%%%%%%%%%%%%%%%%%%%%%%%%
\type{func} becomes : \inframed{\type{(cons set-func (list args-instrument1 args-instrument2 …))}}
\blank[0.4em]
Each \type{args-instrument} of the list is either a single element or either a list itself, depending on the number of parameters required by \type{set-func}.
\blank[0.3em]
Example 5 below, copies patterns for 3 measures and then changes the pitch of the notes in
\blank [0.3em] % the \high below need Y space
 the 2\high{nd} measure.
\blank [0.3em]

This is done using 3 functions that will be seen later :
\startitemize[5,intro][margin=\lilyspace,before=\blank]
\item The \goto{\type{fill} function}[fill] \at{page}[fill] ({\em music}s patterns)
\item The \goto{\type{set-pitch} function}[set-pitch] \at{page}[set-pitch]. It waits for a single parameter, of type {\em music}.
\item The \goto{\type{chords->nmusics} function}[chords->nmusics] \at{page}[chords->nmusics]. It returns a list of \type{n} elements that are just of type …~{\em music}.
\stopitemize

\blank[1.5em]
\underbar{\sc\Words example 5}
\startfiguretext[right][exemple5]{}
{\vbox{\blank[-3em]\externalfigure[exemple05.pdf][wfactor=270, align=flushright]}}
\startlily[space=fixed]
global = { s1*3 
           \bar "|." }
instrus = #'(I II III)

#(init instrus)
\stoplily
\stopfiguretext
\godown[-2em]
\startlily
chords = \relative c' { <b f' gis> <d f b> <c e a> <b d e> }

#(begin
(fill instrus (list #{ r8 e'-. #}
                    #{ r8 c'-. #}
                    #{ a8-> r c'-. r b-. r a-. r #})
              1 4)
(apply-to instrus (cons set-pitch (chords->nmusics 3 chords))
                  2 3)
\stoplily
\blank[1em] %\godown[-1.2em]
\midaligned{ \tfd ------------}
\godown[1.5em]

%%%%%%%%%
\func{x-apply-to}
\syntax {(x-apply-to obj func from-pos1 to-pos1
              / from-pos2 to-pos2 /…)}{} 
Simple shortcut for :
\startmylily
(apply-to obj func from-pos1 to-pos1)
(apply-to obj func from-pos2 to-pos2)
etc…
\stopmylily
The slash \type{/} is optional.\\
A key : \type{obj-start-pos} can optionally specify a starting point that differs from the beginning of the song~:
\startmylily
(x-apply-to obj func pos1 pos2 #:obj-start-pos pos3 …)
\stopmylily
\godown[1.5em]
\midaligned{ \tfd ------------}
\godown[1.5em]

%%%%%%%%%
\func{xchg-music} (shortcut of "e\underbar{xch}ange music")
\syntax {(xchg-music obj1 obj2 from-pos1 to-pos1 / from-pos2 to-pos2 /…)}{}
Copy \type{[from-posn to-posn[} section from \type{obj1} to \type{obj2}, and the one from \type{obj2} to \type{obj1}.\\
The slash \type{/} is optional.
\blank
\midaligned{ \tfd ------------}
\page %%%%%%%%%%%%%%%%%%%%%%%%%%%%%%%%%%%

                      %%%%%%%%%%%%%%%%%%%%%%%%%%%%%%

\subject[agencement]{Manipulating musical elements}
The following functions help manipulating sequential or simultaneous musics, extracted from instruments.
\godown[2.5em]

%%%%%%%%%
\func{em} : from \underbar{e}xtract and \underbar{m}usic, reference function : 
\type+\extractMusic+
\footnote{See {\em DOCS/extractMusic-doc.pdf} at \from[extractMusic]}
\syntax {(em obj from-pos to-pos}{obj-start-pos)}
Extract music in measures range \type{[from-pos to-pos[}. An event will be kept if it begins between theses two limits, and his length will be cut if it lasts after \type{to-pos}.\\
\type{obj} is typically an {\em instrument}, or a list of {\em instrument}s.\\
If \type{obj} is a {\em music} or a {\em music}s list, the \type{obj-start-pos} parameter will inform the function about the position of \type{obj} in the piece (by default : the beginning of the piece).

\type{em} returns a {\em music}s list if \type{obj} is a list, or a {\em music} in the opposite.\\
See an example of use in the following example (function \type{seq}).
\blank[1.2em]
\midaligned{ \tfd ------------}
\godown[1.2em]

%%%%%%%%%
\func{x-em}
\syntax {(x-em pos1 pos2 / pos3 pos4 / ...)}{}
Returns : \type{(list (em obj pos1 pos2) (em obj pos3 pos4) ...)}
\blank[1.2em]
\midaligned{ \tfd ------------}
\godown[1.2em]

%%%%%%%%%
\func{seq}(shortcut of {\em \underbar{seq}uential})
\syntax {(seq musicI musicII musicIII etc…)}{}

Equivalent to : \type+{ \musicI \musicII \musicIII…}+

All arguments are {\em music}s but list of {\em music}s are also supported.\\
\underbar{\sc\Words Example} :
\startmylily
(rm 'clar 12 (seq (em 'flute 12 15)         ; &\em{&Double the flute&}&
                  #{ r2 r4 #}               ; &\em{&Measure 15&}&
                  (em 'violon '(16 -4) 20)) ; &\em{&Double the violin&}&
\stopmylily 
\blank[1.2em]
\midaligned{ \tfd ------------}
\godown[1.2em]

%%%%%%%%%%%%
\func{seq-r}(r like {\em\underbar{r}est})
\syntax {(seq-r . args)}{}

Same as \type{seq} but any number is converted into a rest with that number as duration.\\
The previous example can be written :
\startmylily
(rm 'clar 12 (seq-r (em 'flute 12 15)         ; &\em{&Double the flute&}&
                    2 4                       ; &\em{&Measure 15 : some rests&}&
                    (em 'violon '(16 -4) 20)) ; &\em{&Double le violin&}&
\stopmylily
A dot after the number is possible : \type{4.} means \type+#{ r4. #}+.\\
For 2, 3 dots or more, the digit 2, 3 etc… is added after the dot:\\
\type{4.3} for example means \type+#{ r4... #}+ (3 dots)

\blank[1.2em]
\midaligned{ \tfd ------------}
\page %%%%%%%%%%%%%%%%%%%%%%%%%%%%

%%%%%%%%%
\func{sim} (shortcut of {\em \underbar{sim}ultaneous})
\syntax {(sim musicI musicII musicIII etc…)}{}

Equivalent to : \type+<< \musicI \musicII \musicIII …>>+ 

All arguments are {\em music}s but list of {\em music}s are also supported.\\
See an example in \goto{\type{volta-repeat->skip} function}[volta-repeat->skip], \at{page}[volta-repeat->skip]
\blank[1em]
\midaligned{ \tfd ------------}
\godown[1em]

%%%%%%%%%
\func{split}
\syntax {(split ['(id1 id2 id3...)] music1 music2 music3...)}{}

Equivalent to : \type+\voices id1,id2,id3 ... << music1 \\ music2 \\ music3 ... >> +\\
The {\em id}s list for each voices is optional.\\
The default list is based in the pattern \type+'(1 3 5 ... 6 4 2)+
\blank[1em]
\midaligned{ \tfd ------------}
\godown[1em]


%%%%%%%%%
\func{part-combine}
\syntax {(part-combine musicI musicII)}{}

Equivalent to : \type+\partCombine \musicI \musicII+\\
Both arguments are {\em music}s.
\blank[1em]
\midaligned{ \tfd ------------}
\godown[1em]

%%%%%%%%%
\func{def!}
\syntax {(def! name}{music)}

Equivalent to a Lilypond déclaration : %
\type+name = \music+

\type{name} is an {\em instrument}, or an '{\em instrument}s list.
(\type{def!} is applied to each instruments of the list).\\
\type{music} is a {\em music} or a {\em music}s list.(\type{music1} is associated to \type{instrument1}, \type{music2} to \type{instrument2} etc…)\\
If \type{music} is omitted, the default value is a skip (\type{s1{*}}...) with the same length as  \type{\global}.\\
See example below, in function \type{volta-repeat->skip}.
\blank[1em]
\midaligned{ \tfd ------------}
\godown[1em]

%%%%%%%%%
\func{at}
\syntax {(at pos mus)}{}

Return \type+{ s1*… \mus }+, with \type{s1*…} with a length from beginning of the piece to \type+pos+.

\blank[1em]
\midaligned{ \tfd ------------}
\godown[1em]

%%%%%%%%%
\func{cut-end}
\syntax {(cut-end obj new-end-pos [start-pos])}{}

Cut, at position \type{new-end-pos}, the musics associated with \type{obj}, keeping only the beginning.\\
It is  particularly usefull during building process of \type{\global}, as shown in \goto{addendum I}[addendum1] \at{page}[addendum1].

\blank[1em]
\midaligned{ \tfd ------------}
\page  %%%%%%%%%%%%%%%%%%%%%%%%%%%%

%%%%%%%%%
\index{pos-sub}
\blank[-2em]
\func{volta-repeat->skip}
\syntax {(volta-repeat->skip r . alts)}{}

Returns a \type+\repeat volta [\alternate]+ structure, where each element is a \type{\skip}.\\
The repetitions count is computed from the elements count of \type{alts} (or ignored if empty).\\
All arguments are rational numbers, in the p/q form, with q as a power of two (1 2 4 8…). They indicate the length of each element.
\startmylily (volta-repeat->skip 9 3 5/4) \stopmylily
is equivalent to :
\startmylily \repeat volta 2 s1*9 \alternate { s1*3 s4*5 } \stopmylily \\
Alternatively, arguments can be of type \type{moment}. It allows the use of the internal function \type{pos-sub} which returns a \type{moment} equal to the difference of the 2 positions.\\
For example, \type{(pos-sub 24 13)} returns the length between measure 13 and measure 24 : easy to compute in a 4/4 signature, but more difficult if the section has a lot of measure changes (as \type{\time 7/8} then  \type{\time 3/4} etc …).\\
You can use the \type{def!} function (described \at{page}[def!]), to create a variable containing the various repetitions in the piece :\\
\underbar{\sc\Words Example 5} :
\blank[0.5em]
\startmylily 
(def! 'structure)                ; &\em{&same length as \global&}&  
(rm-with 'structure              ; &\em{&add repetitions&}&  
   5 (volta-repeat->skip 9 3 1)                          ; &\em{&(in &}&4/4&\em{&)&}& 
  29 (volta-repeat->skip (pos-sub 38 29) (* 2 3/4) 3/4)) ; &\em{&(in &}&7/8&\em{& and &}&3/4&\em{&)&}&   
(def! 'global (sim global structure))  ; &\em{&global = << \global \structure >>&}&
\stopmylily
\placefigure[here]{}{\externalfigure[volta-repeat->skip][hfactor=max]}

\midaligned{ \tfd ------------}
\godown[1em]

%%%%%%%%%
\func{mmr}
\syntax {(mmr ratio)}{}
\syntax {(mmr from-bar-num to-bar-num)}{}
Returns a \type{multiMeasasureRest} with a length of either \type{(ly:make-moment ratio)}(for example\\ \type{(mmr 3/4)} for \type+#{ R4*3 #}+, or \type{(mmr 2)} for \type+#{ R1*2 #}+), or either with the length of the music \type{from-bar-num} to \type{to-bar-num} (for example 
\type{(mmr 5 13)} to get a rest that completely fulfils bar 5 to 13). Syntax 2 internally uses the \type{pos-sub} function described above.
\blank[1em]
\midaligned{ \tfd ------------}
\page  %%%%%%%%%%%%%%%%%%%%%%%%%%%%

                      %%%%%%%%%%%%%%%%%%%%%%%%%%%%%%                     
\subject[gerer-voix]{Managing voices (addition, extraction )}
See also {\em chordsAndVoices-doc.pdf} at \from[chord]
\blank[-1em]
%%%%%%%%%
\func{voice} 
\syntax {(voice n [m p …] music)}{}
\index{set-voice}
or : (2\high{nd} equivalent form, to be used with \goto{\type{apply-to}}[apply-to])
\syntax {((set-voice n [m p …]) music)}{}
Extract the voice \type{n} in a music with several simultaneous voices:\\
\godown[0.1em]
\startmylily music = << { a b } \\ { c d } >> \stopmylily
\godown[-0.5em] 
\starttabulate[|l|l|c|l|]
    \NC\myindent\NC {\type{(voice 1 music)}}\NC $\Longrightarrow$ \NC \type+ { a b } +
    \NR
    \NC\myindent\NC {\type{(voice 2 music)}} \NC $\Longrightarrow$  \NC \type+ { c d } +
    \NR
    \NC\myindent\NC {\type{(voice 3 music)}} \NC $\Longrightarrow$  \NC \type+ { c d } +
    \NR
    \NC\myindent\NC { \space \dots} \NC 
\stoptabulate
\godown[0.7em]
If other numbers\type{ m p … }are given, the function returns a list of all voices matching with 
the numbers \type{n m p …}
\startmylily 
(rm '(instru1 instru2) 5 (voice 1 2 music)) 
\stopmylily
is equivalent to :
\startmylily 
(rm 'instru1 5 { a b })
(rm 'instru2 5 { c d })
\stopmylily

\godown[0.2em]
\midaligned{ \tfd ------------}\\
\godown[0.5em]

%%%%%%%%%
\func{replace-voice} 
\syntax {(replace-voice n music repla)}{}
\index{set-replace-voice}
or : (2\high{nd} equivalent form, to be used with \goto{\type{apply-to}}[apply-to])
\syntax {((set-replace-voice n repla) music)}{}
Replaces, in a simultaneous music, the voice \type{n}:\\
\startmylily
music = << { a b } \\ { c d } >>
(replace-voice 2 music #{ f g #})
\stopmylily
returns:\\
\startmylily << { a b } \\ { f g } >> \stopmylily

\godown[0.2em]
\midaligned{ \tfd ------------}\\
\godown[0.5em]

%%%%%%%%%
\func{dispatch-voices}
\syntax {(dispatch-voices obj where-pos music-with-voices}{voices-extra-pos obj-start-pos)} 
\underbar{\sc\Words Example} :
\godown[0.6em]
\startmylily music = << { c2 d } \\ { e2 f } \\ { g2 b } >> \stopmylily 
\godown[0.4em]
The code :
\startmylily (dispatch-voices '(bassoon clarinet (oboe flute)) 8 music) \stopmylily 
will produce, measure 8, the following assignment :
\godown[-0.2em] 
\starttabulate[|l|l|c|l|]
    \NC\myindent\NC {\type{'bassoon}}\NC $\leftarrow$ \NC \type+ { c2 d } +
    \NR
    \NC\myindent\NC {\type{'clarinet}} \NC $\leftarrow$ \NC \type+ { e2 f  } +
    \NR
    \NC\myindent\NC {\type{'oboe}} \NC $\leftarrow$ \NC \type+ { g2 b } +
    \NR
    \NC\myindent\NC {\type{'flute}} \NC $\leftarrow$ \NC \type+ { g2 b } +
    \NR
\stoptabulate 
\godown[-0.2em]
See \in{the \type{rm} function (\at{page}[rm-info])}[rm-info] for the signification of the optional arguments.

\godown[1em]
\midaligned{ \tfd ------------}
\page  %%%%%%%%%%%%%%%%%%%%%%%%%%%%%%%%%%%%%%%%%%%%%%%%%%%%%%%%%

%%%%%%%%%
\hairline
The following functions are all created, at the parameter level, on the same model. Each of them just allows to obtain a particular type of simultaneous music :
\starttabulate[|l|l|c|l|]
    \NC\myindent\NC {\type{add-voice1/add-voice2}}\NC $\rightarrow$ \NC \type+ << \voiceI \\ \voiceII >> +
    \NR
    \NC\myindent\NC {\type{merge-in/merge-in-with}} \NC $\rightarrow$ \NC \type+ << \voiceI \voiceII >> +
    \NR
    \NC\myindent\NC {\type{combine1/combine2}} \NC $\rightarrow$ \NC \type+ \partCombine \voiceI \voiceII +
    \NR
\stoptabulate 
\hairline

\godown[3em]
\func{add-voice1, add-voice2}{}
\syntax{(add-voice1 obj where-pos new-voice}{voice-start-pos to-pos obj-start-pos)}        
\syntax{(add-voice2 obj where-pos new-voice}{voice-start-pos to-pos obj-start-pos)}
The music of each {\em instrument}, is replaced at the \type{where-pos} position with 
\startlily << [existing music] \\ new-voice >>  &\rm{&for&}& add-voice2 \stoplily
and with :
\startlily << new-voice \\ [existing music] >>  &\rm{&for&}& add-voice1&\rm{&.&}& \stoplily

\type{obj} is an {\em instrument} or a list of {\em instrument}s

\type{new-voice} is a {\em music} or a list of {\em music}s.\\
Use \type{voice-start-pos}, if \type{new-voice} begins before \type{where-pos}.\\
Use \type{to-pos} if you want to stop the replacement before the end of \type{new-voice}.\\ 
Use \type{obj-start-pos} if \type{obj} doesn't begin to the beginning of the piece (typically  
measure~1,~see \at{\type{init} function, page}[init]).

\godown[0.2em]
\midaligned{ \tfd ------------}\\
\godown[0.2em]

%%%%%%%%%
\func{merge-in}
\syntax{(merge-in obj where-pos new-voice}{voice-start-pos to-pos obj-start-pos)}

music of \type{obj} is replaced , measure \type{where-pos}, by : 
\startlily << new-voice [existing music] >> \stoplily
For optional parameters, see above (\type{add-voice1}).
\blank
\midaligned{ \tfd ------------}
\godown[1em]

%%%%%%%%%
\func{merge-in-with}
\syntax {(merge-in-with obj pos1 music1 / pos2 music2 / pos3 music3 …)}{}
is a shortcut for :
\startlily 
(merge-in obj pos1 music1)
(merge-in obj pos2 music2)
(merge-in obj pos3 music3)	
…
\stoplily
The slash \type{/} is optionnal
\blank
\midaligned{ \tfd ------------}
\page  %%%%%%%%%%%%%%%%%%%%%%%%%%%%%%%%%%%%%%%%%%%%%%%%%%%%%%%%%

%%%%%%%%%
\func{combine1, combine2}{}
\syntax{(combine1 obj where-pos new-voice}{voice-start-pos to-pos obj-start-pos)}        
\syntax{(combine2 obj where-pos new-voice}{voice-start-pos to-pos obj-start-pos)}
music of each {\em instrument}, is replaced, at  \type{where-pos} position, by :  
\startmylily \partCombine [existing music] \new-voice  &\rm{&for&}& combine2 \stopmylily
and by
\startmylily \partCombine \new-voice [existing music]  &\rm{&for&}& combine1&\rm{&.&}& \stopmylily
\\See \type{add-voice} function in the top of this page, for optional parameters.
\blank[1em]
\midaligned{ \tfd ------------}

                      %%%%%%%%%%%%%%%%%%%%%%%%%%%%%%
\subject[gerer-accords]{Managing chords }
\blank[-1em]
%%%%%%%%%
\func{note}
\syntax {(note n [m p  …] music)}{}
\index{set-note}
or : (2\high{nd} equivalent form, to be used with \goto{\type{apply-to}}[apply-to])
\syntax {((set-note n [m p …]) music)}{}

Extract the n\high{\itxx th} note of each chords (in the same order as in the source file).\\
If other numbers\type{ m p … }are specified, \type{note} will form chords instead, by extracting from original chords, notes matching to these numbers\type{ n m p … }\\
If no match is found, \type{note} returns the last note of the chord.\\ 
\godown[1em]
\underbar{\sc\Words Example} :
\startmylily
music = { <c e g>-\p <d f b>-. }
(note 1 music)   &$\Longrightarrow$& { c-\p d-. }
(note 2 3 music) &$\Longrightarrow$& { <e g>-\p <f b>-. }
(note 4 music)   &$\Longrightarrow$& { g-\p b-. }
\stopmylily
\blank[1em]
\midaligned{ \tfd ------------}
\godown[1em]

%%%%%%%%%
\func{notes+}
\syntax {(notes+ music newnotes1 [newnotes2…])}{}
\index{set-note†}
or : (2\high{nd} equivalent form, to be used with \goto{\type{apply-to}}[apply-to])
\syntax {((set-notes+ newnotes1 [newnotes2…]) music))}{}
Transforms each note of \type{music} into a chord, and inserts in it, the corresponding \type{newnotes}  note.\\
A \type{\skip} in \type{newnotes} leaves the original note unchanged.\\
\godown[1em]
\underbar{\sc\Words Example} :
\startmylily
music = { c'4  b  <g c'>2  c'  c' }
notes = { e  <d f>  e      s   c  }	
(notes+ music notes) &$\Longrightarrow$& { <e c'>4 <d f b> <e g c'>2 c' <c c'> }
\stopmylily
\blank[1em]
\midaligned{ \tfd ------------}
\page %%%%%%%%%%%%%%%%%%%%%%%%%%%%%%

%%%%%%%%%
\func{add-notes} 
\syntax {(add-notes obj where-pos newnotes1 [newnotes2]…[obj-start-pos])}{}
Same as \type{notes+} but applied now to a given position \type{where-pos}\\
\type{obj} can be an {\em instrument}, a list of {\em instrument}s, a {\em music} or a list of {\em music}s.\\
\type{newnotes} are {\em music}s, but if both \type{newnotes1} and \type{obj} are lists, \type{notes+} is applied element to element.\\
See \in{the \type{rm} function (\at{page}[rm-info])}[rm-info] to know the signification of last optional parameter \type{obj-start-pos}.
\blank[2em]
\midaligned{ \tfd ------------}
\godown[2em]

%%%%%%%%%
\func{dispatch-chords}  
\syntax {(dispatch-chords instruments where-pos music-with-chords . args)}{}
\type{dispatch-chords} assigns each note of the chords of a {\em music} to separate parts..\\
\type{instruments} is the list of instruments that receive, at the where-pos position, those parts.\\
\type{music-with-chords} is the {\em music} containing the chords.
The note 1 of a chord is sent to the last item in the list \type{instruments}, then the note 2 to the second to last one etc…\\
The code :
\startmylily
music = { <c e g>4 <d f b>-. }
(dispatch-chords '(alto (tenorI tenorII) basse) 6 music)
\stopmylily
will result, at measure 6, in :
\startmylily
basse   &$\leftarrow$& { c4 d-. }
tenorI  &$\leftarrow$& { e4 f-. }
tenorII &$\leftarrow$& { e4 f-. }
alto    &$\leftarrow$& { g4 b-. }
\stopmylily
The optional args are the same than \in{the \type{rm} function (\at{see page}[rm-info])}[rm-info] 
\blank[2em]
\midaligned{ \tfd ------------}
\godown[2em]

%%%%%%%%%
\func{reverse-chords}  
\syntax {(reverse-chords n music}{strict-comp?)}
\index{set-reverse}
or : (2\high{nd} equivalent form, to be used with \goto{\type{apply-to}}[apply-to])
\syntax {((set-reverse n [strict-comp?]) music)}{}
Reverse \type{n} times chords contained in \type{music}.\\
The displaced note is octavated as many times as necessary to make its pitch higher (lower if \type{n<0}) than the note preceding it.\\
The optional parameter \type{strict-comp?} proposes either, when set to \type{#t}, the comparison: {\em strictly} higher ({\em strictly} lower for \type{n<0}), or, when set to \type{#f}, the comparison: higher (lower) or {\em equal}.\\
By default, \type{strict-comp?} is set to \type{#f} for \type{set-reverse} and to \type{#t} for \type{reverse-chords} !
\blank[2em]
\underbar{\sc\Words Example} (in absolute pitch mode) :
\page %%%%%%%%%%%%%%%%%%%%%%%%%%%%%%%%%%%%
\startmylily
                   music    &\hbox to 1.3em {\\$=$\\}\hbox to 0.1em {\\}&  { <c e g>   <c g e'>    <c e c'> }
(reverse-chords 1 music)    &$\Longrightarrow$&  {   <e g c'>  <g e' c''>  <e c' c''> }
(reverse-chords 2 music)    &$\Longrightarrow$&  {     <g c' e'> <e' c'' g''><c' c'' e''> }

(reverse-chords 0 music)    &$\Longrightarrow$&  {       <c e g>   <c g e'>  <c e c'> }
(reverse-chords -1 music)   &$\Longrightarrow$&  {    <g, c e>  <e, c g>  <c, c e> }
(reverse-chords -2 music)   &$\Longrightarrow$&  { <e, g, c><g,, e, c><e,, c, c> }

(reverse-chords 1 music #f) &$\Longrightarrow$&  {   <e g c'>  <g e' c''>  <e c' c'> }
\stopmylily
\blank
\midaligned{ \tfd ------------}
\godown[1.2em]

%%%%%%%%%
\func{braketify-chords}  
\syntax{(braketify-chords obj)}{}
Adds bracket in chords containing at least 2 notes and not linked in previous chord by a tilde~\type{~} \\
This function extends the \type{\braketifyChords} function defined in {\em copyArticulations.ly} accepting also as parameter, a list of {\em music}s, an {\em instrument}, or a list of {\em instrument}s.
\blank
\midaligned{ \tfd ------------}
\blank

                      %%%%%%%%%%%%%%%%%%%%%%%%%%%%%%                     
\subject[gerer-accords-voix]{Managing chords and voices together}
\blank[0.3em] 
The following functions are all compatible with \goto{\type{apply-to}}[apply-to].
\blank[-0.5em] 

%%%%%%%%%
\func{treble-of}
\syntax {(treble-of music)}{}

Extract in first voice, the last note of each chord.

\blank
\midaligned{ \tfd ------------}
%%%%%%%%%
\func{bass-of} 
\syntax {(bass-of music)}{} 

Extract in last voice, the first note of each chord.

\blank
\midaligned{ \tfd ------------}
%%%%%%%%%
\func{voices->chords} 
\syntax {(voices->chords [n] music)}{}

Replaces all {\em simultaneous musics} of \type{music} by a {\em sequential musics} with \type{n}  notes chords (If omitted, \type{n} defaults to 2).\\
By default, the 1\high{st} note of a chord matchs to the last voice and so on, but notes order
in chords can be customized by setting \type{n} as a list of numbers.\\
\startlily
music = << { e' g' } { c' d' } { a b } >>
(voices->chords 3 music)        &\rm{results in}& { <a c' e'> <b d' g'> }
(voices->chords '(3 2 1) music) &\rm{results in}& { <a c' e'> <b d' g'> } &\rm{(idem)}&
(voices->chords '(2 1 3) music) &\rm{results in}& { <c' e' a> <d' g' b> }
\stoplily
The results obtained are exactly the same with :
\startlily
music = << { e'4. g'8 } \\ { c'4 d' } \\ { a8 b4. } >>
\stoplily
and the rhythm will be extracted from voice 1, i.e. \type+{ 4. 8 }+ \\
\page %%%%%%%%%%%%%%%%%%%%%%%%%%%%%%%%%%%%
On the other hand, with :
\startlily
music = \voices 1,3,2 << { e'4. g'8 } \\ { c'8 d'4. } \\ { a b } >>
\stoplily
the 1\high{st} chord will be preceded by a long sequence of \type+\override+ or \type+\set+\\
The function \goto{\type{set-del-events}}[set-del-events] can then be used, to keep only the notes.
\startlily
((set-del-events 'OverrideProperty 'PropertySet)(voices->chords 3 music))
\stoplily

\blank
\midaligned{ \tfd ------------}\\
\godown[1.5em]

%%%%%%%%%
\func{chords->voices} 
\syntax {(chords->voices [n] music)}{}
The function use the function \goto{\type{note}}[note] (\at{page}[note]) and the function \goto{\type{split}}[split] (\at{page}[split]).\\
It is equivalent to:
\startlily 
(split (note n music) 
       (note (- n 1) music)
       … 
       (note 1 music))
\stoplily
By default, \type+n = 2+\\
\type+n+ is converted by the \goto{\type{split}}[split] function into a list of {\em id}s for each voice. However, a list of numbers can be directly specified, taking into account that the 1\high{st} note of a chord corresponds to the last voice.
\startlily
music = { <a c' e'> <b d' g'> }
(chords->voices 3 music) &\rm{and}& (chords->voices '(1 3 2) music) &\rm{result in} :& 
\voices 1,3,2 << { e' g' } \\ { c' d' } \\ { a b } >>
\stoplily

\blank
\midaligned{ \tfd ------------}\\
\godown[1.5em]

%%%%%%%%%
\func{chords->nmusics} 
\syntax {(chords->nmusics n music)}{}
\index{set-chords->nmusics}
or : (2\high{nd} equivalent form, to be used with \goto{\type{apply-to}}[apply-to])
\syntax {((set-chords->nmusics n) music)}{}

Transform a sequence of chords in a {\em list} of\type{ n }{\em music}s\\
\blank
For:\type+ music = {<e g c'> <d f b> <c e g c'>}+\\
the \type{chords->nmusics} function give the following list :

\blank
\hbox{{\hspace[lilyindent]}
\starttable[|c|l|]
 \NC n \VL liste \NC\SR
 \HL
 \NC 1 \VL \type+{e d c}+ \NC\FR
 \NR
 \NC 2 \VL \type+{g f e}{e d c}+ \NC\MR
 \NR
 \NC 3 \VL \type+{c' b g}{g f e}{e d c}+ \NC\MR
 \NR
 \NC 4 \VL \type+{c' b c'}{c' b g}{g f e}{e d c}+ \NC\LR
\stoptable
}
\blank[1.5em]
See a use of \type{chords->nmusics} at \goto{example 5}[exemple5] of \at{page}[exemple5].
\blank[1.5em]
\midaligned{ \tfd ------------}
\page %%%%%%%%%%%%%%%%%%%%%%%%%%%%%%

                     %%%%%%%%%%%%%%%%%%%%%%%%%%%%%%

\subject[gerer-hauteur]{Managing pitch of notes}
\blank[-2em]
%%%%%%%%%
\func{rel}
\syntax {(rel [n] music)}{}
returns : \type{\relative} {\em pitch} \type{\music}\\
\midaligned{{\em pitch} as the central \type{c'}, transposed by \type{n} octaves.\type{    }}

\hbox{{\hspace[lilyindent]}
\starttable[s0|rs0|l|]
 \NC \type{(rel -2 music)} \Longrightarrow \NC\JustLeft\type{ \relative c, \music} \NC\FR
 \NR
 \NC \type{(rel -1 music)} \Longrightarrow \NC\JustLeft\type{ \relative c \music} \NC\MR
 \NR
 \NC \type{(rel music)}    \Longrightarrow \NC\JustLeft\type{ \relative c' \music}  \% {\em by défault:} \type{n=0} \NC\MR
 \NR
 \NC \type{(rel 1 music)}    \Longrightarrow \NC\JustLeft\type{ \relative c'' \music} \NC\MR
 \NR
 \NC \type{(rel 2 music)}  \Longrightarrow \NC\JustLeft\type{ \relative c''' \music} \NC\LR
\stoptable
} % end of \hbox

An extended syntax is possible. See \goto{\type{octave} function}[octave], \at{page}[octave].

\blank[1em]
\midaligned{ \tfd ------------}\\

%%%%%%%%%
\func{set-pitch} (reference function : \type+\changePitch+)
\syntax {((set-pitch from-notes) obj)}{}

Replace pitch of notes in \type{obj} by those in \type{from-notes}.
To use typically with \goto{\em apply-to}[apply-to]. See \goto{example 5}[exemple5] at \at{page}[exemple5].
\blank[0.6em]
\midaligned{ \tfd ------------}\\

%%%%%%%%%
\func{set-transp}
\syntax {((set-transp octave note-index alteration/2) obj [obj2 [obj3 …]])}{}
\syntax {((set-transp func) obj [obj2 [obj3 …]])}{}
Apply the Lilypond scheme function \type{ly:pitch-transpose} to each pitch of \type{obj}, with a {\em "delta-pitch"} parameter equal to :
\startitemize[joinedup,intro]
  \nop either the return value of \type{(ly:make-pitch octave note-index alteration/2)} (syntax 1)
  \nop either the return value of the \type{func(p)} function (syntax 2).(\type{p} current pitch to transpose).
\stopitemize
\blank[0.2em]
The \type{obj} parameters are {\em music}s, {\em instrument}s or a list of one of these 2 types.\\
The function returns the transposed {\em music}, or a list of transposed {\em music}s\\
\type{set-transp} is  compatible with \in{\type{apply-to}}[apply-to] and can be used as 
follows :
\startlily
#(let((5th (set-transp 0 4 0))}      ; &\em{\small &4 notes above = a fifth&}&
      (3rd (set-transp 0 2 -1/2))    ; &\em{\small &like from c to ees&}&
      (enhar (set-transp 0 1 -1)))   ; &\em{\small &from c to deses = enharmony&}&
   (rm all 67 (5th (em all 11 23)))  ; &\em{\small &[11-23] is copied at 67 to the fifth&}&
   (rm '(AclarI AclarII) 1 (3rd cl1 cl2))  ; &\em{\small &concert pitch transposed in A&}&
   (apply-to 'saxAlto enhar 10 15)   ; &\em{\small &set [10-15] in the enharmonic tone&}&
\stoplily
The function \type{maj->min} presented now, uses syntax 2 to adapt the transposition interval around the modal notes (degree III and VI) of the original major key. 
\startfiguretext[right,fit][set-transp-func]{}
 {\vbox{\blank[-0.2em]\externalfigure[set-transp-func-en][wfactor=345, align=flushright]}}
 {\vbox{\blank[0.5em]
\setupinterlinespace[big]
\type{'II} and \type{'III} are transposed from C major to A and C minor.}}
\stopfiguretext
\setupinterlinespace[small]
\godown[-0.55em]
The function \type{maj->min} is defined as follows:
\page %%%%%%%%%%%%%%%%%
\startmylily
#(define (maj->min from-pitch to-pitch)  ;&\em{\small& returns the function lambda &}&
   (let ((delta (ly:pitch-diff to-pitch from-pitch))
         (special-pitches (music-pitches ;&\em{\small& see scm/music-functions.scm &}&
           (ly:music-transpose #{ dis e eis gis a ais #} from-pitch))))
     (lambda(p) (ly:make-pitch           ;&\em{\small& returns the delta pitch &}&
       0 
       (ly:pitch-steps delta) 
       (+ (ly:pitch-alteration delta) ;&\em{\small& the interval varies according to p &}&   
          (if (find (same-pitch-as p 'any-octave) special-pitches)
            -1/2  0))))))      ;&\em{\small& same-pitch-as is defined in checkPitch.ly &}&
\stopmylily
All that's left is to choose which \type{to-pitch} parameter to apply to \type{'II} and \type{'III}:
\startmylily 
(apply-to 'II (set-transp (maj->min  #{ c' #} #{ a #})) 1 8) 
(apply-to 'III (set-transp (maj->min #{ c' #} #{ c' #})) 1 8))            
\stopmylily

\godown[0.8em]
\midaligned{ \tfd ------------}
\godown[0.2em]

%%%%%%%%%
\func{octave}
\syntax {(octave n obj)}{}
\godown[-0.3em]
or : (2\high{nd} equivalent form, to be used with \goto{\type{apply-to}}[apply-to])
\godown[-0.3em]
\index{set-octave}
\syntax {((set-octave n) obj)}{}

Basically, \type{octave} is a simple shortcut to the function \type{(set-transp n 0 0)}, where \type{n} can be positive (upward transposition) or negative (downward transposition).\\
However, like the \type{rel} and \type{octave+} functions, it has an extended syntax.\\
Here are some possibilities.
\godown[0.5em]

\underbar{1\high{st} case}: putting a theme in different octaves, for instruments of different tessitura.
\blank[1mm]
\startmylily
(rm '(vlI vlII va (vc db)) 18 (octave 2 1 0 -1 theme))
\stopmylily 
\blank[1mm]
The function returns the list \type{((octave 2 theme)(octave 1 theme)} etc …\type{)}\\
Note that the cello and the double bass receive the same music: \type{(octave -1 theme)}
\godown[0.5em]

\underbar{2\high{nd} case} : putting in a specified octave, several musics at the same time.
\blank[1mm]
\startmylily
(rm '(instruI instruII instruIII instruIV) 18 (octave 1 m1 m2 m3 m4))
\stopmylily 
\blank[1mm]
All musics \type{m1 m2 m3 m4} are transposed by one octave.
\godown[0.5em]

\underbar{3\high{rd} case} : great mix !
\blank[1mm]
\startmylily
(rm '(vlI vlII va (vc db)) 18 (octave 2 m1 1 m2 m3 -1 m4))
\stopmylily 
\blank[1mm]
\type{m1} is transposed 2 octaves up, \type{m2} and \type{m3} are transposed : 1 octave up, and \type{m4} is transposed : 1 octave down.

\godown[0.2em]
\midaligned{ \tfd ------------}
\godown[1em]

%%%%%%%%%
\func{octavize} 
\godown[-0.1em]
\syntax {(octavize n obj from-pos1 to-pos1 [/ from-pos2 to-pos2 /…])}{} 
\type{octavize} transpose by \type{n} octaves the {\em instrumen}t (or the list of {\em instrument}s)\type{ obj}, between the positions \type{[from-pos1 to-pos1]}, \type{[from-pos2 to-pos2]}, etc…

\godown[0.7em]
\midaligned{ \tfd ------------}
\godown[0.2em]

%%%%%%%%%
\func{octave+}
\godown[0em]
\syntax {(octave+ n music)}{}
\godown[-0.2em]
or : (2\high{nd} equivalent form, to be used with \goto{\type{apply-to}}[apply-to])
\godown[-0.5em]
\index{set-octave†}
\syntax {((set-octave+ n) obj)}{}
\godown[-0.3em]
Shortcut of \type!(notes+ music (octave n music))! (see \in{\type{notes+} \at{page}{}[notes+]}[notes+]) but without doubling articulations.\\
\type{octave+} has the same extended syntax as \type{octave} (see above) and \type{rel}.

\godown[0.7em]
\midaligned{ \tfd ------------}
\page %%%%%%%%%%%%%%%%%%%%%%%%%%%%%%

%%%%%%%%%
\func{add-note-octave} 
\syntax {(add-note-octave n obj from-pos1 to-pos1 [/ from-pos2 to-pos2 /…])}{} 

Apply the previous \type{(octave+ n music)} function to each \type+[from-pos to-pos]+ section.

\godown[1em]
\midaligned{ \tfd ------------}
\godown[2em]

%%%%%%%%%
\hairline
The 2 following functions : \type{fix-pitch} and \type{pitches->percu} are more specifically designed for percussion. They put a bridge between notes with pitch and percussion notes.
\hairline
\godown[2em]

%%%%%%%%%
\func{fix-pitch}
\syntax {(fix-pitch music pitch)}{}
\syntax {(fix-pitch music note-index)}{}
\syntax {(fix-pitch music octave note-index alteration)}{}
\index{set-fix-pitch}
Sets all the notes to pitch \type{pitch} (syntax 1) or \type{(ly:make-pitch -1 note-index 0)} (syntax 2), or finally \type{(ly:make-pitch octave note-index alteration)} (syntax 3).\\
These 3 lines are equivalent:
\startmylily[space=fixed]
(fix-pitch music #{ c #})
(fix-pitch music 0)
(fix-pitch music -1 0 0)
\stopmylily
The corresponding \goto{\type{apply-to}}[apply-to] function \type{((set-fix-pitch …) music)} takes these same pitch parameters.
\blank[0.2em]
\midaligned{ \tfd ------------}
\godown[1em]

%%%%%%%%%
\func{pitches->percu} 
\syntax {(pitches->percu music percu-sym-def . args)}{}
Converts notes to percussion-type notes.\\
\type{args} is a sequence of a pitch following by a percussion symbol.\\
For each note of \type{music}, the function searches for the percussion symbol corresponding to the pitch of this note. If none is found, the default symbol \type{percu-sym-def} is taken.\\
Then this percussion instrument is assigned to the \type{'drum-style} property of the note.\\
Each group of \type{args} can optionally be separated by a slash \type{/}\\ 
Finally, note that any number \type{n} is transformed into \type{(ly:make-pitch -1 n 0)} by the function, as for syntax 2 of the previous function \type{fix-pitch}
\blank[0.5em]
\underbar{\sc\Words exemple 6}
\startfiguretext[right][exemple6]{}
{\externalfigure[exemple06.pdf][wfactor=150, align=flushright]}
\startlily[space=fixed]
music = << 
   { e8 e e e e e e e} \\ 
   { c4 d8 c c4 d8 c } >>
\stoplily
\stopfiguretext
\godown[-2em]
\startlily[space=fixed]
percu = #(pitches->percu music 'hihat /
                       #{ c #} 'bassdrum /  ;&\em{\small& or : 0 'bassdrum /&}&
                       #{ d #} 'snare)      ;&\em{\small& or : 1 'snare)&}&
\new DrumStaff \drummode { \percu }
\stoplily
\godown[1em]
\midaligned{ \tfd ------------}
\page %%%%%%%%%%%%%%%%%%%%%%%%%%%%%%

%%%%%%%%%
\func{set-range} (see : \type{correct-out-of-range} in {\em checkPitch.ly})
\syntax {((set-range range) music)}{} 

\type{range} is a sequence of 2 notes : \type+#{ c, c'' #}+ or a two-tone chord: \type+#{ <c, c''> #}+\\
Transposes to the right octave, all notes out of \type{range}. The function allows you to adjust the score to the tessitura of an instrument, for example.\\
Can be used with \goto{\type{apply-to}}[apply-to].

\godown[1em]
\midaligned{ \tfd ------------}
\godown[2em]

%%%%%%%%%
\func{display-transpose} 
\syntax {(display-transpose music amount)}{} 

Visually moves notes from \type{amount} positions up or down. The midi datas are untouched.

\godown[1em]
\midaligned{ \tfd ------------}
\godown[1em]

%%%%%%%%%
\hairline
The \type{cp} function presented now, takes its name from \underbar{c}hange \underbar{p}itch. It therefore allows you to modify the pitch of the notes of a piece of music without affecting its rhythm, but also to modify the rhythm of a piece of music without affecting the pitch of the notes. This is the reason why it will be part of the following section: rhythm patterns.\\
This remark also concerns the function \type{cp-with}
\hairline
\godown[2em]  

\subject[utiliser-pattern]{Using «patterns»}
\blank[-1em] 

%%%%%%%%%

\func{cp} : {\em rhythm} pattern (reference function is \type+\changePitch+%
\footnote{See {\em changePitch-doc.pdf} at \from[changePitch]})
\index{cp1}
\index{cp2}
\syntax {(cp [keep-last-rests?] pattern[s] music[s])}{}
or : (2\high{nd} equivalent form, to be used with \goto{\type{apply-to}}[apply-to])
\index{set-pat}
\syntax {((set-pat pattern [keep-last-rests?]) obj)}{}

\type{cp} is basically equivalent to \type+\changePitch \pattern \music+\\
It returns a {\em music} when \type{pattern} and \type{music} are {\em music}s, and a list of {\em music}s, if one of those parameters are a list of {\em music}s or {\em instrument}s.\\
If \type{pattern} ends by rests, the optional parameter \type{keep-last-rests?} indicates whether they should also be included after the very last note.\\\type{keep-last-rests?} defaults to  \type{#t} for \type{cp} and to \type{#f} for \type{set-pat}.\\
2 \type{cp} shortcuts have been defined :
\startmylily
(cp1 obj)   &$\Longrightarrow$&   (cp patI obj)
(cp2 obj)   &$\Longrightarrow$&   (cp patII obj)
\stopmylily
See \goto{\type{tweak-notes-seq}} [tweak-notes-seq] (\at{page} [tweak-notes-seq]) for an example of use of the shortcut \type{cp1}
\blank[1em]
\midaligned{ \tfd ------------}
\page %%%%%%%%%%%%%%%%%%%%%%%%%%%%%%%%%%                

%%%%%%%%%
\func{cp-with} 
\syntax {(cp-with obj pos1 notes1 [pos2 notes2 [pos3 notes3…]])}{}
Replaces at position \type{pos}, the notes of \type{obj} with those of \type{notes}.
The original rhythms of \type{obj} remain unchanged, only the pitches are modified. The articulations are mixed.\\
A slash / after each note parameter may help to clarify the code visually.\\
Like \goto{\type{rm-with}}[rm-with], the scheme function \type{delay} can be used to retrieve music modified in a previous section.
\blank[0.3em]
\midaligned{ \tfd ------------}
\godown[0.8em]

%%%%%%%%%
\func{ca} : {\em articulations} pattern (reference function is \type+\copyArticulations+
\footnote{See \from[copyArticulations-LSR] for the use of \type{\copyArticulations}})
\syntax{(ca pattern[s] music[s])}{}
or : (2\high{nd} equivalent form, to be used with \goto{\type{apply-to}}[apply-to])
\index{set-arti}
\syntax{((set-arti pattern) obj)}{}

Copies articulations from \type{pattern} to \type{music}, and returns \type{music}.\\
If at least one of these 2 arguments is a list (a list of musics or a list of instruments),
the function returns a list of musics.\\
It is possible to use in \type{music}s , the functions defined in {\em copyArticulations.ly}:\\\type{\notCopyArticulations} (shortcut \type{\notCA}), \type{\skipArti}, \type{\nSkipArti} et \type{\skipTiedNotes}.
\blank[0.8em]
\midaligned{ \tfd ------------}
\godown[0.7em]


%%%%%%%%%
\func{fill-with} : {\em musics} pattern
\syntax {(fill-with pattern from-pos to-pos)}{}

Repeat the \type{pattern} music the number of times necessary to fill the \type{[from-pos to-pos]} interval exactly, eventually cutting off the last copy.\\
Returns the resulting music, or a list of these musics if \type{pattern} is a list of musics.
\blank
\midaligned{ \tfd ------------}
\godown[0.8em]

%%%%%%%%%
\func{fill} : {\em musics} pattern
\syntax {(fill obj pattern from-pos to-pos . args)}{} 

Equivalent of \type{(rm obj from-pos music)} with
\startmylily music = (fill-with pattern from-pos to-pos) \stopmylily
The following syntax is possible:
\startmylily (fill obj pat1 from1 to1 / [pat2] from2 to2 / [pat3] from3 to3 …) \stopmylily
If a \type{pat} parameter is omitted, the one from the previous section is retrieved.\\
See \goto{example 5}[exemple5] \at{page}[exemple5].
\blank[0.5em]
\midaligned{ \tfd ------------}
\godown[0.8em]

%%%%%%%%%
\func{fill-percent} : {\em musics} pattern
\syntax {(fill-percent obj pattern from-pos to-pos . args)}{} 

Same as function \type{fill} above, but produces \type{\repeat percent …} musics instead.
\startfiguretext[right][fill-percent]{}
{\vbox{\blank[-0.8em]\externalfigure[fill-percent.pdf][wfactor=200, align=flushright]}}
\startlily
(fill-percent 'I 
   #{ c'4 d' e' f' #} 1 4)
(fill-percent 'II 
   #{ c'4 d' #} 1 4)
\stoplily
\stopfiguretext
%\godown[-3em]
\blank
\midaligned{ \tfd ------------}\\

%%%%%%%%%
\func{tweak-notes-seq} : {\em notes} pattern
\syntax {(tweak-notes-seq n-list music)}{} 
or : (2\high{nd} equivalent form, to be used with \goto{\type{apply-to}}[apply-to])
\index{set-tweak-notes-seq}
\syntax {((set-tweak-notes-seq n-list) music)}{}
\type{music}  is a music with notes in it.\\
\type{n-list} is an integers list. Each number \type{n} represents the n\high{th} note extracted from \type{music}.\\
\type{tweak-notes-seq} returns a sequential music by replacing each number of \type{n-list} with the corresponding note. When the last number is reached, the process starts again at the beginning of the list of numbers, but increasing it by the largest number in the list. The process stops when there are no more notes to match in \type{music}.
\startmylily
(tweak-notes-seq '(1 2 3 2 1) #{ c d e | d e f | e f g #}) 
& $\Longrightarrow$ & { c d e d c
      d e f e d
      e f g f e }
\stopmylily
In \type{n-list}, a number \type{n} can be replaced by a pair \type{(n . music-function)}.\\
\type{music-function} is then applied to the note \type{n}. It must take a music as parameter and return a music. Typically, this function is \goto{\type{set-octave}}[octave].\\
The following example uses this function, in combination with the \goto{\type{cp1}}[set-pat] shortcut of the \type{set-pat} function.

%\blank[0.7cm]
\startfiguretext[right][exemple7]{}
{\vbox{\blank[-0.5em]\externalfigure[exemple07.pdf][wfactor=235, align=flushright]}}
\underbar{\sc\Words exemple 7}
%\blank[1cm]
\startlily[space=fixed]
patI = { r8 c16 c c8 c c c }
#(rm 'instru 1 (cp1 
   (tweak-notes-seq 
     `(1 2 3 (1 . ,(set-octave +1)) 3 2) 
     (rel 1 #{ c e g | a, c e | f, a c | g b d #}))))
\stoplily
\stopfiguretext
\midaligned{ \tfd ------------}\\

%%%%%%%%%%
\func{x-pos} : {\em bar numbers} pattern
\syntax {(x-pos n-from n-to [pos-pat [step]])} {}
\syntax {((x-pos [pos-pat [step]]) n-from1 n-to1 [n-from2 n-to2...])} {}

The \type{n-from} and \type{n-to} parameters are bar numbers (some {\em integers}).\\
\type{pos-pat} is a {\em position}s \footnote {positions are defined \in{in the \quotation{music  positions} paragraph, \at{page}[positions_musicales].}[positions_musicales]} list, with a letter, generally n, instead of bar numbers.\\
\type{x-pos} converts this list, replacing n (the letter) by the bar number \type{n-from} and increasing it recursively by \type{step} units, as long as this value remains strictly inferior to \type{n-to}.\\
In syntax 2, \type{x-pos} successively applies the same pattern \type{pos-pat} and \type{step} to each of the \type{n-from}/\type{n-to} pairs. \\
By default, \type{pos-pat} = \type{'(n)}, \type{step} = 1.
\page %%%%%%%%%%%%%%%%%%%%%%%%%%%%%%%%%%%%%%%%%
The following table shows the list obtained with different values:
\startmylily
(x-pos 10 14)              & $\Longrightarrow$ & '(10 11 12 13)
(x-pos 10 14 '(n (n 4)))   & $\Longrightarrow$ & '(10 (10 4) 11 (11 4) 12 (12 4) 13 (13 4))
(x-pos 10 14 '(n (n 4)) 2) & $\Longrightarrow$ & '(10 (10 4) 12 (12 4))
(x-pos 10 13 '(n (n 4)) 2) & $\Longrightarrow$ & '(10 (10 4) 12 (12 4))
(x-pos 10 12 '(n (n 4)) 2) & $\Longrightarrow$ & '(10 (10 4))
\stopmylily
\blank
\type{x-pos} can be used with \type{x-rm} for example, in conjunction with the scheme function \type{apply}:
\blank[0.6em]
\reference[exemple8]{exemple8}
\starthanging[right]
{\vbox{\blank[5em]\externalfigure[exemple08.pdf][wfactor=305, align=flushright]}}
\underbar{\sc\Words example 8}
\startlily
global = {s1*12 \bar "|."}
music = { e'2 f' | g' f' | e'1 }

cls = #'(I II)
#(init cls)

#(begin
  (rm cls 10 music)
  (apply x-rm 'clII #{ c'8 c' c' #} (x-pos 10 13 '((n 8)(n 2 8)))))
\stoplily %\blank[2.1cm]%
\stophanging
\godown[1em]

                      %%%%%%%%%%%%%%%%%%%%%%%%%%%%%%                                                
\subject[ajouter-text]{Adding text and musical quotations }
\blank[-2em]
%%%%%%%%
\func{txt}
\syntax {(txt text [dir [X-align [Y-offset]]])}{}

\type{text} is a {\em markup} \\
\type{dir} is the {\em direction} of \type{text} : %
\type{1} (or \type{UP}), \type{-1} (or \type{DOWN}), or by default \type{0} (automatic).\\
\type{X-align} is the {\em self-alignment-X} property value of \type{text} : \type{-1} by default.\\
\hbox{{\hspace[lilyindent]}
\starttable[|c|c|]  
\NC \type{X-align} \VL text alignment \NC\SR
\HL
\NC -1 or \type{LEFT}\VL to left \NC\FR
\NC 1 or \type{RIGHT}\VL to right \NC\MR
\NC 0 or \type{CENTER} \VL center \NC\LR
\stoptable 
}\\
\type{Y-offset} is the {\em Y-offset} property value of \type{text} : 0 by default\\
The function returns a zero-length {\em skip}.\\
\underbar{\sc\Words Exemple} :
\startmylily (txt "Hello" UP 0 -2) \stopmylily
is equivalent to :
\startmylily
s1*0 -\tweak self-alignment-X #CENTER 
     -\tweak Y-offset #-2 
     ^"Hello"  % &\em{&^ = UP &}&
\stopmylily
Note that setting one of the optional parameters \type{dir}, \type{X-align} or \type{Y-offset} to the value \type{#f}, has the same effect as omitting this parameter: its corresponding property is not modified.
\blank[1em]
\midaligned{ \tfd ------------}
\godown[1em]

%%%%%%%%%
\func{adef}
\syntax {(adef music [text [dir [X-align [Y-offset]]]])}{}

Formats \type{music} with cue notes, like in a {\em \quotation {a def}} section . A text can be added   with the same arguments as the previous \type{txt} function.
\page %%%%%%%%%%%%%%%%%%%%%%%%%%%%%%%%%%%%%%%%%%%%%%%
\reference[exemple9]{example9}
\underbar{\sc\Words Example 9} :
\godown[0.5em]
\setupnarrower[left=\lilyindent]
\defineparagraphs[mypar][n=2,before={\blank[0.2em]\startnarrower},after={\blank[0.8em]\stopnarrower},distance=0cm]
\setupparagraphs[mypar][1][width=0.31\textwidth]
%\setupparagraphs[mypar][2][width=\textwidth]
%%%%
\startmypar
\godown[1.4em]
Consider the fol-
\crlf lowing violin:
\mypar %%%%%
\mbox{\hspace[lilyindent]\externalfigure[exemple09-violon.pdf][wfactor=240]}
%\placefigure[here][]{}{\externalfigure[exemple09-violon.pdf][wfactor=240]}
\stopmypar
%%%%
\startmypar
\godown[0.2em]
and a flute begin-
\crlf ning bar 4~:
\mypar %%%%%
\godown[0.5em]
\startmylily (rm 'fl 4 (rel #{ f'4 g a b | c1 #})) \stopmylily
\stopmypar
%%%%
\startmypar
\godown[1.2em]
The following code~:
\mypar %%%%%
\startmylily
(add-voice2 'fl 3 
       (adef (em vl 3 4) "(violon)" DOWN))
(rm 'fl 4 (txt "play" UP))       
\stopmylily
\stopmypar\godown[0.5em]
%%%%
\startmypar
\godown[1.8em]
will produce the flute~:
\mypar %%%%%
\mbox{\hspace[lilyindent]\externalfigure[exemple09-flute-en.pdf][wfactor=240, align=flushleft]}
\stopmypar
\godown[0.5em]
The difference in size of a {\em \quotation {a def}} section from the current size is \type+adef-size = -3+. You can redefine \type+adef-size+ as you wish. For example, it can be:
\startmylily(define adef-size -2)\stopmylily
If we want to have, in the example above, the text: "(violin)" at the normal size, we must replace this text by the following {\em markup}:
\startmylily (markup (#:fontsize (- adef-size) "(violon)")) \stopmylily
\blank[1.2em]
\midaligned{ \tfd ------------}
\godown[-0.8em]
                      %%%%%%%%%%%%%%%%%%%%%%%%%%%%%%                                              
\subject{Adding dynamics}
\godown[2em]
%%%%%%%%
\func{add-dynamics}
\syntax {(add-dynamics obj pos-dyn-str)}{}

\type{obj} is a {\em music}, an {\em instrument}, or a list of {\em instrument}s.\\
\type{pos-dyn-str} is a {\em string}  "…", composed by a sequence of position-dynamics, separated by a slash~\type{/} (the slash is mandatory here).\\
The function analyzes the string \type{pos-dyn-str} and returns a code of the form:
\startmylily
(rm-with obj pos1 #{ <>\dynamics1 #} / pos2 #{ <>\dynamics2 #} /…)
\stopmylily
For list positions, the \type{'} character can be omitted: 
\type{'(11 4 8)} $\Longrightarrow$ \type{(11 4 8)}.\\
For dynamics, all backslashes~\type+\+ {\em must} be removed. Direction symbols, on the other hand, \type+-^_+ are allowed. Several dynamics are separated with a space.
\godown[1.6em]
\underbar{\sc\Words Example}:
\godown[0.6em]
Taking the violin from the previous example 9, the following code:
\godown[-0.4em]
\startlily
(add-dynamics 'vl "1 mf / 2 > / 3 p cresc / (4 2) ^f")
\stoplily
\godown[-0.4em]
will result in:
\godown[0.3em]
\mbox{\hspace[lilyindent]\externalfigure[exemple09-violon-nuances.pdf][wfactor=265, align=flushleft]}
\blank[2em]

- A position followed by no dynamic tells the function to search and delete the previous dynamic that would occur at the same {\em moment}.

\page %%%%%%%%%%%%%%%%%%%%
- It is possible to specify adjustments of the position \type{X} and \type{Y} of a dynamic \type{dyn} by the following basic syntax (it will be adapted in most cases): \type{dyn#X#Y}.\\
Something like: \type{mf#1#-1.5} will result to:
\godown[-0.4em]
\startlily <>-\tweak self-alignment-X #1 -\tweak extra-offset #'(0 . -1.5) -\mf \stoplily
\godown[-0.4em]
To replace the {\em zero} of the first element of the \type{extra-offset} pair, we can also put a third parameter between the other two. The general syntax then becomes:
\godown[0.5em]
\startmylily dyn#val1#val3#val2 \stopmylily
\godown[0.5em]
and it results to:
\godown[0.5em]
\startmylily <>-\tweak self-alignment-X val1 -\tweak extra-offset #'(val3 . val2) -\dyn \stopmylily 
\godown[0.3em]
A \type{val} value can be omitted but the number of \type{#} characters must match to the 
index 1,2 or 3~:\\
\hbox{{\hspace[lilyindent]}
\starttable[s0|ls2|c|l|]
 \NC \type{#val}   \NC \Longrightarrow \NC\JustLeft\type{val1} : \type{self-alignment-X val}\NC\FR
 \NC \type{##val}  \NC \Longrightarrow \NC\JustLeft\type{val2} : \type{extra-offset #'(0 . val)}\NC\MR
 \NC \type{##val#} \NC \Longrightarrow \NC\JustLeft\type{val3} : \type{extra-offset #'(val . 0)}\NC\MR
 \NC \type{##valA#valB} \NC \Longrightarrow \NC\JustLeft\type{val3,val2} : \type{extra-offset #'(valA . valB)}\NC\LR
\stoptable
}\\ % end of \hbox
\godown[0.5em]
- Regardless of these placement adjustments induced by the \type{\tweak} command, the \type{add-dynamics} function allows very precise placement of dynamics by judicious choice of its associated musical position. However, if it is easy, for example, to insert a dynamic at the position \type{'(3 64)}, there is a problem if a fourth starts at bar \type{3} because it will be cut at the 64\high{th} beat !\\
It would therefore be wise to create a special separate voice for the instrument \type{instru}, named \type{instruDyn} for example, made up only of \type{skip}s and which would receive all the \type{instru} dynamics.\\
Then simply combine that voice with the voice of notes and with \type{global}.The example at the beginning of the paragraph will become~:
\godown[-0.4em]
\startlily
(def! 'vlDyn)  ; & \at{see page}{}[def!]. &
(add-dynamics 'vlDyn "1 mf / 2 > / 3 p cresc / (4 2) ^f")
…
\new Staff { << \global \vlDyn \vl >> }
\stoplily
\godown[-0.4em]
Note that this is identical to the traditional way of proceeding, except that here there is no need to make calculations to find the adequate duration of the \type{skip}s between 2 dynamics. It's {\em arranger.ly} that takes care of it.\\
Also note that {\em arranger.ly} introduces a \goto{\type{sym-append} function}[sym-append], which is particularly well suited to the creation of these special voices. See the given example \at{at page}{}[sym-append], precisely with voices dedicated to dynamics.\\
Finally, note that this method makes it possible to insert dynamics in \type{tuplet}s:\\
\startnarrower
- A {\em forte} for the 2nd 8th note of a triplet in bar 5, can be obtained with \type{"(5 12) f"} \footnote{There are 12 triplet 8th notes in a whole note.},\\
- the 3rd 8th note can be obtained by \type{"(5 12 12) f"} or \type{"(5 6) f"}.\\
\stopnarrower
The syntax with fractions can only be used, in \type{add-dynamics}, through variables to be included in the string parameter:
\startlily
#(define frac 1/12)
#(add-dynamics 'vlDyn "(5 frac) f / (5 (* 2 frac) p") ;&\tfx\em{& '(5 1/12) and '(5 2/12)&}&
\stoplily
- Dynamics within a \type+\grace+ section require, on the other hand, a particular syntax.\\
To indicate them within the code, we will use the character : (colon), immediately followed by the duration \type{(8 16 ...)} of the \type{skip} "carrying" eventually the dynamic within the section~\type+\grace+.\\
\vtop{\hbox{
 \hspace[lilyindent]
 \starttable[lT|c|lT|]
    %% ConTeXt "eats" spaces after \grace => we need 2 \type
    \type{"p:8"} \NC will result in \NC \type+{ \grace+\type+ { s8\p } <> }+ \FR
    \NR
    \type{":16 mf:16"} \NC will result in \NC \type+{ \grace+\type+ { s16 s16\mf } <> }+ \MR
    \NR
    \type{"<:16 :16*2 f"} \NC will result in \NC \type+{ \grace+\type+ { s16\< s16*2 } <>\f }+ \LR    
    \NR
  \stoptable}
}

\page %%%%%%%%%%%%%%%
The character \type+#+ for dynamic position tweaks may be used in conjunction, but it must be placed after (and without spaces in) the \type+\grace+ section\\
\startnarrower
\type+"mf:8#1" + will result in \type+ { \grace+\type+ { s8-\tweak self-alignment-X #1 \mf } <> }+
\stopnarrower
\startfiguretext[right][exemple]{}
{\vbox{\blank[2.5em]
       \hbox{\externalfigure[grace-dynamics.pdf][wfactor=110, align=flushright]
             \hspace[big]}}}
\underbar{\sc\Words Exemple} :
\godown[-0.2em]
\startlily[space=fixed]
#(begin           ; &\em{&dynamics in a \grace section&}&
(def! '(dyn1 dyn2 dyn3))  ; &\em{&dedicated voices s1*&\dots}&  
(add-dynamics 'dyn1       ; &\em{&a simple cresc&}&
  "1 p / (1 2) <:16 :16*2 f")      
(add-dynamics 'dyn2   ; &\em{&dynamics without tweaks&}&
  "1 p / (1 2) :16 mp:16 mf:16 f") 
(add-dynamics 'dyn3   ; &\em{&dynamics with tweaks&}&
  "1 p / (1 2) :16 mp:16#1.3#-1.2
                   mf:16#0#-0.6 
                   f#-0.2#-0.6"))
\score { <<
     \new Staff $(sim global instru1 dyn1)
     \new Staff $(sim global instru2 dyn2)
     \new Staff $(sim global instru3 dyn3)
   >> }
\stoplily
\stopfiguretext

- In case a \type+\grace+ section is the 2\high{nd} parameter of a \type+\afterGrace+ command, a special syntax is required for the 1\high{st} parameter ("the main note") .\\
It will suffice, here, to precede the \type+\grace+ section with the rhythmic value of the 1\high{st} parameter, which will be marked with a double character \type+::+\\
\startmylily
\afterGrace s4\f { s16\p s }
f::4 p:16:16     &\leftarrow\em{\tfx& the possible nuance before :: the rhythmic value after&}&
\stopmylily
The optional fraction of the \type+\afterGrace+ command is obtained as follows:
\startmylily
\afterGrace 15/16 s4\f { s16\p s }
f::4:15:16 p:16:16
\stopmylily
We will make sure to match the fraction of the music and the nuances.
\startfiguretext[right][example]{}
{\vbox{\blank[5.85em]
       \hbox{\externalfigure[afterGrace-dynamics.pdf][wfactor=180, align=flushright]
             \hspace[big]}}}
\underbar{\sc\Words Example}:
\godown[-0.2em]
\startlily[space=fixed]
musicI = { r2 r4 \afterGrace d4-> { es16( e f fis }
           g4->) r r2 } % &\em{\tfx&fraction &}&3/4 &\em{\tfx&by def &}&
musicII = { r2 r4 \afterGrace 15/16 d4-> { es16( e f fis }
            g4->) r r2 } % &\em{\tfx&fraction&}& 15/16
#(rm all 1 (rel 1 musicI musicII))

#(begin
(def!'(dyn1 dyn2))
(add-dynamics 'dyn1  ; &\em{\tfx&fraction &}&3/4
  "(1 2.) f::4 p:16 <:16 :8 / 2 f")
(add-dynamics 'dyn2  ; &\em{\tfx&fraction &}&15/16
  "(1 2.) f::4:15:16 p:16 <:16:8 / 2 f")
)

\stoplily
\stopfiguretext

\hairline
The following functions, \type{assoc-pos-dyn}, \type{extract-pos-dyn-str}, \type{instru-pos-dyn->music} and \type{add-dyn}, are attempts to further simplify the management of dynamics, in particular by avoiding 1) the redundant informations to provide for instruments having the same dynamics at the same moments, and 2) to solve the problem of duplicate dynamics when, in orchestral scores, 2 instruments shares the same staff.
\hairline
\godown[3em]

\func{assoc-pos-dyn}
\syntax {(assoc-pos-dyn pos-dyn-str1 instru1 / pos-dyn-str2 instru2 /…)}{}
The \type{pos-dyn-str}s are strings as defined in the above \goto{\type{add-dynamics}}[add-dynamics] function.\\
\type{instru} is either a single \type{instrument} or a list of \type{instrument}s.\\
The function returns an {\em associated-list} consisting of {\em pairs} \type{'(pos-dyn-str . instru)}.\\  
The slashes~\type{/} are optional.
\godown[1em]
\underbar{\sc example} :
\blank[0.8em]
\startmylily
vls = #'(vlI vlII) 
horns = #'(hornI … hornIV) 
all = #'(fl oboe cl …)
assocDynList = #(assoc-pos-dyn            
  "1 p" 'hornI / "5 mf" vls / "25 f / (31 4) < " horns / 
  "33 ff / 35 decresc / 38 mf" all …)
\stopmylily
\blank[0.5em]
Dynamics for a single instrument, can then be extracted by setting \type{assocDynList} as last parameter of the 2 functions \type{extract-pos-dyn-str} or \type{instru-pos-dyn->music}.
\blank[1em]
Finally, please note that a string like \type{"1 f / 3 mf / 5 p"} can also be entered as a list: \\
\type{'("1 f" "3 mf" "5 p")}. The \at{addendum 3 page} [set-dyn] shows a use of this automatic formatting.
\godown[1.3em]
\midaligned{ \tfd ------------}
\godown[1.3em]

%%%%%%%%
\func{extract-pos-dyn-str}
\syntax {(extract-pos-dyn-str extract-code assoc-pos-dyn-list)}{}
\type{assoc-pos-dyn-list} is the association list created with the \in{\type{assoc-pos-dyn}} [assoc-pos-dyn] function above.\\
The function \type{extract-pos-dyn-str} returns a {\em pos-dyn-str}, as defined in \in{\type{add-dynamics}}[add-dynamics]. It is the concatanation of all {\em pos-dyn-str}s whose associated {\em instrument}s return \quotation{true} to the \type{extract-code} predicate.
\blank[0.7em]
Here's how the \type{extract-code} predicate works:\\\\

- \type{extract-code} is either a single {\em instrument}, or a list of {\em instrument}s with one of the following three logical operators as the first element: \type{'or 'and 'xor}\\
For a single {\em instrument}, \type{extract-code} returns \quotation{true} when the list of  instruments associated with a particular {\em pos-dyn-str}, contains this instrument.\\
For 2 instruments, it depends on the operator:\\
\godown[1.3em]
\hbox{{\hspace[lilyindent]}
\starttable[|c|c|]  
\NC \type{extract-code} \VL associated list \NC\SR
\HL
\NC \type{'a} \VL contains \type{'a} \NC\FR
\NC \type{'(and a b)} \VL contains \type{'a} {\bold \underbar and} \type{'b} \NC\MR
\NC \type{'(or a b)} \VL contains \type{'a} {\bold \underbar or} \type{'b} \NC\MR
\NC \type{'(xor a b)} \VL contains \type{'a} but {\bold \underbar not} \type{'b} \NC\LR
\stoptable  
}\\
\page %%%%%%%%%%%%%%%%%%%%%%%%%
\underbar{\sc{Example}} :
\startmylily
horns = #'(hornI hornII hornIII)
assocDynList = #(assoc-pos-dyn            
    "1 p" 'hornI / "5 mf <" '(hornI hornII) / "6 ff > / 7 !" horns)   
%% &\em{&Simple extraction&}&
#(extract-pos-dyn-str 'hornIII assocDynList) 
   => "6 ff > / 7 !"
%% &\em{&Extraction with operator&}&
#(instru-pos-dyn-str '(or hornI hornII) assocDynList)
   => "1 p / 5 mf < / 6 ff > / 7 !" 
#(instru-pos-dyn-str '(xor hornI hornII) assocDynList)
   => "1 p"
#(instru-pos-dyn-str '(and hornI hornII) assocDynList)
   => "5 mf < / 6 ff > / 7 !"
\stopmylily
\blank[0.5em]
- More than 2 items to an operator are allowed. The third element is combined with the result of the operation of the first two.
\startmylily
'(and a b c) = '(and (and a b) c)
\stopmylily
\blank[0.5em]
- A list of {\em instrument}s can be made up of sub-lists. If a sub-list does not begin with an operator, its items are copied to the higher-level list.
\godown[1.5em]
\midaligned{ \tfd ------------}
\godown[1.5em]
%%%%%%%%
\func{instru-pos-dyn->music}
\syntax {(instru-pos-dyn->music extract-code assoc-pos-dyn-list)}{}
Same as \type{extract-pos-dyn-str} above, but the return string is converted using \type{add-dynamics},  into a {\em music} in the form:
\startmylily
{ <>\p s1*4 <>\mf s1*29 <>\ff }
\stopmylily
\godown[1.5em]
\midaligned{ \tfd ------------}
\godown[1.5em]

%%%%%%%%
\func{add-dyn}
\syntax {(add-dyn extract-code)}{}
\type{(add-dyn extract-code)} is a macro (shortcut) of the function \type{instru-pos-dyn->music} above, which avoids specifying the last parameter \type{assoc-pos-dyn-list}. It is defined as follows:
\startmylily
#(define-macro (add-dyn extract-code)
   `(instru-pos-dyn->music ,extract-code assocDynList))
\stopmylily
So this macro will only work if you have defined an \type{assocDynList} variable:
\startmylily assocDynList = #(assoc-pos-dyn...) \stopmylily
\godown[1em]
\hairline
Additional informations on \type{assocDynList} is provided \at{page}{ in the addendum 3}[complément-assocDynList].
\hairline
\page %%%%%%%%%%%%%%%%%%%%%%%%%

\subject{Managing tempo indications, keys and marks}  %%%%%%%%%%%%%%%%%%%%%%%%%%%%%%
\godown[0.4em]
The following functions are used in the \in{addendum I about \type{\global}, \at{page}[addendum1]:}[addendum1]
\godown[2.5em]

%%%%%%%%
\func{metronome}
\syntax {(metronome mvt note x [txt [open-par [close-par ]]])}{}
Returns an {\em markup} equivalent to that provided by the \type{\tempo} function.\\
- \type{mvt} is a {\em markup} indicating the movement of the piece. For example : "Allegro"\\
- \type{note} is a {\em string} representing a note value: "4." for a dotted fourth, "8" for a eigth…\\
- \type{x} represents either a metronomic tempo if \type{x} is an {\em integer}, or as for the previous argument, a {\em string} representing a note value. See the example of the  \type{tempos} function below.\\
- Optionally, the \type{txt} argument allows to add, after the metronomic indication, a text such as \quotation{env} or \quotation{ca.}.\\
- Using the arguments \type{open-par} and \type{close-par}, one can change (or delete, by putting \type{""}) the opening and closing parentheses surrounding the metronome indication.
\godown[2em]
\midaligned{ \tfd ------------}
\godown[2em]

%%%%%%%%
\func{tempos}
\syntax {(tempos [obj] posA mvtA [spaceA] / posB mvtB [spaceB] / …)}{}
Insert in \type{obj} and at the position \type+pos+, the metronome indication \type+\tempo txt+.\\
Si \type{obj} is omitted, the indication is inserted in \type+\global+\\
If a \type{space} number is specified, the \type{txt} {\em markup} is moved horizontally by $+$ or $-$ \type{space} units to the right or left.\\
Slashes / are optional.
\blank[0.8em]
\underbar{\sc\Words Example} :
\godown[0.7em]
\startmylily
(tempos 1 "Allegro" / 50 (metronome "Andante" "4" 69) /
      100 (metronome "Allegro" "4" "8") -2 ; &\tfx\em{&will be moved 2 units to the left&}&
      150 (markup #:column ("RONDO" (metronome "Allegro" "4." "4")))
\stopmylily
\godown[2em]
\midaligned{ \tfd ------------}
\godown[2em]

%%%%%%%%
\func{signatures}
\syntax {(signatures posA sig-strA [/] posB sig-strB ...)}{}
Inserts rhythmic signatures in \type{\global}, at positions indicated by the {\em pos} %
arguments.\\
A {\em sig-str} argument is made up of the arguments of a \type{\time} command (basically a fraction), placed between 2 quotation marks "\dots".
\godown[0.5em]
\startmylily
(signatures 1 "3/4"
           10 "3,2 5/8"
           20 "4/4")      
& $\Longrightarrow$ &
(rm-with 'global 1 #{ \time 3/4 #}
                10 #{ \time 3,2 5/8 #}
                20 #{ \time 4/4 #})
\stopmylily
\godown[0.5em]
Each group of arguments can be separated by a slash /.
\godown[2em]
\midaligned{ \tfd ------------}
\page %%%%%%%%%%%%%%%%%%%

\func{keys}
\syntax {(keys [obj] posA key-mode-strA [/] posB key-mode-strB [/] ...)}{}
Inserts \type{\key} commands in \type{obj}, at positions \type{pos}.\\
If \type{obj} is not specified, the command is inserted in \type+\global+\\
An argument \type{key-mode-str}, of type {\em string} "\dots", is made up of the same 2 arguments as for the function \type{\key}: 1\high{st} argument the tone, 2\high{nd} the mode.\\
The mode argument can be omitted. The default mode is \type{\major}.\\
The backslash \type+\+ before the mode is obtained either by doubling it (~\type{\\major} instead of \type{\major}~), \dots~or by omitting it (~\type{major} instead of \type{\major}~).\\

\startmylily
(keys 1 "f minor"
     20 "c major"
     40 "f") 
& $\Longrightarrow$ &
(rm-with 'global 1 #{ \key f \minor #}
                20 #{ \key c \major #}
                40 #{ \key f \major #})
\stopmylily
Each group of arguments can be separated by a slash /.
\godown[1.2em]
\midaligned{ \tfd ------------}
\godown[1em]

\func{marks}
\syntax {(marks [obj] posA [/] posB [/] ...)}{}
Inserts a \type{\mark \default} command at positions \type{pos}, in \type{'global} or in \type{obj} if specified.
\godown[1.2em]
\midaligned{ \tfd ------------}
\godown[-1.3em]

                      %%%%%%%%%%%%%%%%%%%%%%%%%%%%%%                                              
\subject{Manipulating lists}
In addition to the basic functions \type{cons} and \type{append} of {\em\sc GUILE}, we may need  some of the following functions.
\godown[2.1em]
%%%%%%%%
\func{lst} (\type{lst} and also \type{flat-lst}) 
\syntax {(lst obj1 [obj2…])}{}
\index{flat-lst}

\type{obj1}, \type{obj2…} are {\em instrument}s or list of {\em instrument}s.\\
Return a list of all {\em instrument}s given in parameters.
\godown[0.7em]
\underbar{\sc\Words Example} :
\startmylily
tps = #'(tpI tpII)
horns = #'(hornI hornII) 
tbs = #'(tbI tbII)
brass = #(lst tps horns tbs 'tuba)
\stopmylily
The last instruction is equivalent to:
\startmylily
brass = #'(tpI tpII hornI hornII tbI tbII tuba)
\stopmylily
\type{lst} keeps the sub lists untouched.\\
With this instruction:
\startmylily
tps = #'(tpI (tpII tpIII))
\stopmylily
the result would be:
\startmylily
brass = #'(tpI (tpII tpIII) hornI hornII tbI tbII tuba)
\stopmylily
If this is not the expected result, we can use the function \type{flat-lst} (same syntax), which returns a list composed only of {\em instrument}s, whatever the depth of the lists given in parameters.
\godown[1em]
\midaligned{ \tfd ------------}
\page %%%%%%%%%%%%%%%%%%%

%%%%%%%%%
\func{lst-diff}
\syntax {(lst-diff mainlist . tosubstract)}{}
Remove from \type{mainlist} the {\em instrument}s specified in \type{tosubstract}.\\
\type{tosubstract} is a sequence of {\em instrument}s or lists of {\em instrument}s.
\godown[0.8em]
\midaligned{ \tfd ------------}
\godown[0.5em]

%%%%%%%%%
\func{zip}
\syntax {(zip x1 [x2…])}{}
\type{x1}, \type{x2…} are standard lists (not circular, predicate \type{proper-list?}).
The function redefines the function \type{zip} of {\em\sc guile}, allowing the addition of all the elements of the biggest lists. The original function \type{zip} of {\em\sc guile} has been renamed \type{guile-zip}.
\startmylily
(guile-zip '(A1 A2) '(B1 B2 B3)) & $\Rightarrow$ & '((A1 B1) (A2 B2))
      (zip '(A1 A2) '(B1 B2 B3)) & $\Rightarrow$ & '((A1 B1) (A2 B2) (B3))
\stopmylily
If the following lists and music have been defined: 
\startmylily
tps = #'(tpI tpII tpIII)
clars = #'(clI clII clIII) 
saxAltos = #'(altI altII)
music = \relative c' { <c e g> <d f b> }
\stopmylily
The following code:
\startmylily
(dispatch-chords (zip tps clars saxAltos) 6 music)
\stopmylily
 will produce bar 6 :
\startmylily
'(tpI clI altI)    & $\leftarrow$ & { g' b' }
'(tpII clII altII) & $\leftarrow$ & { e' f' }
'(tpIII clIII)     & $\leftarrow$ & { c' d' }
\stopmylily
\blank[0.7em]
\midaligned{ \tfd ------------}
\godown[-1.7em]
                      %%%%%%%%%%%%%%%%%%%%%%%%%%%%%%
                                                                                                                      
\subject{Various functions}
\blank[-2em]   
                                   
%%%%%%%%
\func{sym-append}
\syntax {((sym-append sym [to-begin?]) instru[s]}{}
Make a symbol name by adding the symbol \type{sym} to the end of an instrument name (suffix).\\
If \type{to-begin?} is set to \type{#t}, \type{sym} becomes a prefix (pasted at the beginning). \\
This function has to be applied to an {\em instrument} or a list of {\em instrument}s.\\
By associating it to the function \in{\type{def!} \at{at page}{}[def!]}[def!], one can automatically create musics of the form \type+{s1*…}+, with same length as the piece.\\
A typical use is putting all dynamics of an instrument in a separate voice:
\godown[0.4em]
\startmylily
all = #'(oboeI oboeII clarinet violinI violinII viola cello)       
#(let ((dyn-append (sym-append 'Dyn)))   ; &\tfx\em{& 'instru => 'instruDyn&}&
   (def! (dyn-append all))   ;; &\tfx\em{&declaration and initialization of oboeIDyn, oboeIIDyn …&}&
   (add-dynamics 'clarinetDyn "1 p / 4 f …")  ;; &\tfx\em{& adds dynamics&}&
   (add-dynamics '(oboeIDyn oboeIIDyn) "2 p < / 4 f …")
  …)
\stopmylily
\godown[0.3em]
In the separate parts or the score, we'll put~:
\blank[0.3em]
\startmylily
\new Staff << \global \oboeI \oboeIDyn >>
\new Staff << \global \oboeII \oboeIIDyn >>
\new Staff << \global \clarinet \clarinetDyn >> …
\stopmylily
\blank[0.3em]
To lighten the \type{\new Staff} handwriting, one may want to push automation much further.
This is done as an example by the \goto{\type{instru->music}}[instru->music] function in \at{addendum 2, page}[instru->music].
\godown[1em]
\midaligned{ \tfd ------------}
\page %%%%%%%%%%%%%%%%%%%

%%%%%%%%
\func{set-del-events}
\syntax {(set-del-events event-sym . args)}{}
Deletes all events with the name\footnote {An event name begins with a capital letter and ends with \quotation{Event}. Example: {\em 'SlurEvent}} \type{event-sym} \\
Several events can be specified, consecutively or as a list.\\
Thus, the list named \type{dyn-list}, defined in {\em "chordsAndVoices.ly"} as follows:
\blank[0.3em]
\startmylily
#(define dyn-list '(AbsoluteDynamicEvent CrescendoEvent DecrescendoEvent))
\stopmylily
\blank[0.3em]
makes it possible, used with the \type{set-del-events} function, to erase all the dynamics of a portion of music and possibly to replace them by another:
\startmylily
#(let((del-dyn (set-del-events dyn-list))
   (apply-to 'trumpet del-dyn 8 12)
   (add-dynamics 'trumpet "8 p / 10 mp < / 11 mf"))
\stopmylily
\blank[0.7em]
\midaligned{ \tfd ------------}
\godown[1em]

%%%%%%%%
\func{n-copy}
\syntax {(n-copy n music)}{}
\index{set-ncopy}
or : (2\high{nd} equivalent form, to be used with \goto{\type{apply-to}}[apply-to])
\syntax {((set-ncopy n) music)}{}
Copy \type{music} \type{n} times.
\blank[0.1em]
\midaligned{ \tfd ------------}
\godown[1.3em]

%%%%%%%%
\index{index->string-letters}
\func{def-letters}
\syntax {(def-letters measures [index->string][start-index][show-infos?])}{}
The function associates letters with the bar numbers of the \type{measures} list. It is particularly suitable when \type{Score.markFormatter} is of the form \type{#format-mark-[…]-letters}.\\
The following 3 parameters for \type{measures} are optional and differ only in their type.\\
\type{index->string} is a callback function returning a {\em string}, and taking an {\em index} as parameter (a positive integer). The index is incremented by 1 with each call, starting with the value of the \type{start-index} parameter (\type{0} if \type{start-index} is not specified).\\
By default, \type{index->string} is the internal function \type{index->string-letters} that returns the corresponding capital letter(s) to their index in the alphabet, but skips the letter~\quotation{\type{I}}~:
\blank[0.3em]
\hbox{\hspace[lilyindent]"\type{A}"…"\type{H}" then "\type{J}"…"\type{Z}" then "\type{AA}"…"\type{AH}" 
then "\type{AJ}"…"\type{AZ}" etc…}
\blank[0.3em]
The instruction: \type {#(def-letters '(9 25 56 75 88 106))} gives the following matches:\\
\midaligned{\hbox{\hspace[lilyindent]
\starttable[s0|cs1|c|ls2|]
 \NC \type{A} \NC $\Longrightarrow$ \NC measure \type{9} \NC\FR
 \NC \type{B} \NC $\Longrightarrow$ \NC measure \type{25} \NC\MR
 \NC \type{F} \NC $\Longrightarrow$ \NC measure \type{106} \NC\MR
 \NC \type{G} \NC $\Longrightarrow$ \NC <error> \NC\LR
\stoptable
\hspace[threelilyindent]
\starttable[|cs1|c|ls2|]
 \NC \type{(+ A 2)} \NC $\Longrightarrow$ \NC measure \type{11} \NC\FR
 \NC \type{'(A 4 8)} \NC $\Longrightarrow$ \NC <error> \NC\MR
 \NC \type{`(,A 4 8)} \NC $\Longrightarrow$ \NC position \type{'(9 4 8)} \NC\MR
 \NC \type{(list (+ A 2) 4 8)} \NC $\Longrightarrow$ \NC position \type{'(11 4 8)} \NC\LR
\stoptable
\hspace[lilyindent]
}} % end of \hbox
If a letter was already defined before calling \type{def-letters}, the function prepends the character \quotation{\type{_}} to the letter. This is especially necessary for letters \type{X} and \type{Y}, which have \type{0} and \type{1} as associated value in {\em Lilypond}. These 2 letters will thus become {\em always} \type{_X} and \type{_Y}. A message will warn the user of the change, except if we include \type{#f} in the options (parameter \type{show-infos?}):
\godown[0.5em]
\startmylily #(def-letters '(9 25 …) #f) \stopmylily
\godown[0.7em]
\midaligned{ \tfd ------------}
\page %%%%%%%%%%%%%%%%%%%

%% set-frag is not ready enough (bugs remains). Perhaps in a future version of arranger.ly
                      %%%%%%%%%%%%%%%%%%%%%%%%%%%%%%                                                                 
%\subject{"Baliser" une musique}
%\godown[0.3em]
%\underbar{Rappel}
%\blank[0.56em]
%Le fichier {\em addAt.ly} permet de travailler avec des balises qui permettent de répérer un fragment  particulier d'une musique. Deux fonctions y sont définies : \type{\anchor} (balise) et \type{\musicAt}.
%\startlily
%music = \relative c' {
%  \anchor #'A  c4 c c c
%  \anchor #'B  d d d d
%  \anchor #'C  e e e e
%  }
%\musicAt #'A  %%&\em{& => { c c c c } &}&
%\musicAt #'B  %%&\em{& => { d d d d } &}&
%\musicAt #'C  %%&\em{& => { e e e e } &}&
%\stoplily
%Noter que chaque \type{\anchor} re-initialise le mode \type{\relative} à \type{c'} avec la fonction \type{ \resetRelativeOctave}~: tous les \type{\anchor}s sont donc relatif au do central.\\
%{\em arranger.ly} étend la capacité de balisage\\
%%%%%%%%%
%\func{set-frag}
%\syntax {((set-frag music) sym)}{}

                      %%%%%%%%%%%%%%%%%%%%%%%%%%%%%%                                                                 
\subject{Compiling a score section}
\godown[1.6em]
%%%%%%%%
\func{show-score}
\syntax {(show-score from-pos to-pos)}{}
Insert in \type+\global+, \type+\set Score.skipTypesetting = ##t+ or \type+##f+, in order to compile (and show) the music of the score, only between the positions \type+from-pos+ and \type+to-pos+ (useful for large scores).
\blank
\midaligned{ \tfd ------------}
\godown[-0.5em]                     

                     %%%%%%%%%%%%%%%%%%%%%%%%%%%%%%                                              
\subject{Exporting your instruments}
\godown[1.6em]                                        
%%%%%%%%
\func{export-instruments}
\syntax {(export-instruments instruments filename}{overwrite?)}

\type{instruments} is the {\em instrument}s list to export.\\
\type{filename} is the filename of the current path, in which the export will be carried out.\\
The function produces a classic {\em ly} file with statements of the form:
\startmylily
instrument-name = { music … }
\stopmylily
(Notes will be written in absolute mode).\\
If \type{filename} already exists, the instrument definitions will be added at the end of the file, unless \type{overwrite?} is set to \type+#t+: the old version is then deleted!
\godown[0.8em]
\hairline
This function is still in an experimental state! Proceed with caution.
\hairline
\godown[1em]
In the current state of the function, a line break occurs after each measure.\\
However, some events, such as multi-measure silences, are not split into multiple measures:  \type{R1*5} remains \type{R1*5} and not \type+{R1 R1 R1 R1 R1}+.\\
For a simultaneous music \type{<<}\dots\type{>>}, the line break is made for each element and with an indentation.\\
Sequential musics are mixed as much as possible into each other to minimise the resulting code.
\page %%%%%%%%%%%%%%%%%%%%%%%%%%%%%%%%%%%%%%%%

%%%%%%%%%%%%%%%%%%%%%%%%%%%%%
\starttitle[
  reference=addendum1,
  title={-ADDENDUM I-\\BUILDING \type+\global+ WITH \quotation{arranger.ly}},
  list={ADDENDUM I : BUILDING \type+\global+~~~~~~~~~}
  ]

%[addendum1]{ADDENDUM\\CONSTRUIRE \type+\global+ AVEC «arranger.ly»}

\type+\global+ is generally rather tiresome to enter because you have to calculate "by hand" the duration that separates 2 events (between 2 \type+\mark\default+ for example).\\        
Here's how "{\em arranger.ly }" can make the encoding easier, on a 70 bars piece, containing measure changes, key changes, tempos etc…
\startlily
global = { s1*1000 }                  %%&\em\tfx{& a long length is provided &}&
#(init '())                           %%&\em\tfx{& Instruments list initially empty =>&}&
     %%&\em\tfx{& the positions take into account previous timing insertions.  &}&
     %%&\em\tfx{& ( \global is re-analysed each time. ) &}&
#(begin                                ;;&\em\tfx{& Builds \global&}&
(signatures 1 "3/4" 10 "5/8" 20 "4/4") ;;&\em\tfx{& First, time signatures&}&
(cut-end 'global 70)                   ;;&\em\tfx{& Cuts what's beyond&}&
(keys 1 "d minor" 20 "bes major" 30 "d major") ;;&\em\tfx{& Key signatures&}&
(tempos                                ;;&\em\tfx{& Tempo indications&}&
   1 (metronome "Allegro" "4" 120) /
  10 (metronome "" "8" "8") 2 /        ;;&\em\tfx{& 2 unit shift to the right&}&                         
  20 (metronome "Allargando" "4" "4.") 2.5 /
  30 "Piu mosso" -4 /
  60 (markup #:column ("FINAL" (metronome "Allegro vivo" "4" 200))))
(marks 10 20 30 40 50 60)              ;;&\em\tfx{& Marks&}&
(x-rm 'global (bar "||") 20 30 60)     ;;&\em\tfx{& Bars (\bar )&}&
(rm-with 'global 1 markLengthOn        ;;&\em\tfx{& Miscellaneous &}& 
                 ...          
                 70 (bar "|."))       ;;&\em\tfx{& …the final touch&}&
)                                         %%&\em\tfx{& End \global&}&
%%&\em\tfx{& The list of instruments can now be initialized.&}&
#(init '(test)) %%&\em\tfx{& List not empty = fixed metric: any new timing event will be ignored&}&              

\layout {
  \context { \Score
    skipBars = ##t
    \override MultiMeasureRest.expand-limit = #1
    markFormatter = #format-mark-box-letters
  }
}
\new Staff { << \global \test >> }

\stoplily

%\startfiguretext[middle][exemple7]{}
{\externalfigure[addendum-1.pdf][wfactor=fit]}
%\stopfiguretext
\midaligned{\underbar{\sc\Words example 10}}
\stoptitle

\starttitle[
  reference=addendum2,
  title={-ADDENDUM II-\\GETTING ORGANIZED},
  list={ADDENDUM II : GETTING ORGANIZED}
  ]


Here are some ideas for organisation when creating an arrangement for a large orchestral ensemble. Some functions are suggested here, but please note that they are {\em not} part of {\em arranger.ly}. Their définitions have been copied in the file: \type{addendum-functions.ly} in the \type{arrangerDoc-sources} directory of \type{arranger.ly} project.\\

{\tfa\bf→} {\bf Files structure}

\hbox{{\hspace[lilyindent]}
\starttable[s0|ls2|c|c|]
 \NC files \NC usage \NC \type{\include}\SR
 \HL
 \NC init.ily \NC \type{global = {…}} and (\type{init all})\NC"arranger.ly"\NC\FR
 \NC NOTES.ily \NC instruments filling\NC "init.ily" and at end of file: "dynamics.ily" \NC\MR
 \NC dynamics.ily \NC \type{assocDynList = …} \NC - \NC\MR
 \NC SCORE.ly \NC the main score \NC "NOTES.ily" \NC\MR
 \NC parts/instru.ly \NC separate parts \NC "../NOTES.ily" \NC\LR
\stoptable
}\\ % end of \hbox

%%%%%%%%%%%
{\tfa\bf→} {\bf Instrument in separate part vs. instrument in main score.}\\
You may want some of the settings of an instrument to vary when it is edited in a separate part, or in a score. Here's how to get conditional source code.\\

You can place, at the head of each separate parts, the following instruction:
\startmylily #(define part 'instru) ;;&\em\tfx{& the name of the instrument (a symbol)&}&\stopmylily
and at the head of the score:
\startmylily #(define part 'score) \stopmylily
Then add, in the {\em init.ily} file for example, the following function \type{part?} :
\startmylily 
#(define (part? arg) (and (defined? 'part)
                          (if (list? arg)(memq part arg)
                                         (eq? part arg))))
\stopmylily
The instruction \type{(if (part? 'instru) val1 val2)}, or \type{(if (part? '(instruI instruII)) val1 val2)}, can then be used in the code.\\
In the following example, the text will be left-aligned in the score and right-aligned in the euphonium part:\type{ (rm 'euph 5 (txt "Bring out !" UP (if (part? 'score) LEFT RIGHT))) }\\
%%%%%%%%%%
\blank[1em]
\index{obj->music}
\index{instru->music}
\reference[instru->music]{}
{\tfa\bf→} {\bf Separate parts: a \type{instru->music} function}\\
Prerequisite: having \type{assocDynList} be defined (in file {\em dynamics.ily})\\
\type{instru->music} uses \type{obj->music}, a function returning the music associated with an instrument\footnote{\type{(obj->music 'clar)} returns \type{clar}}, and the function \type{make-clef-set} (defined in \type{scm/parser-clef.scm} file, in the {\em Lilypond} directory): \type{make-clef-set} is the scheme equivalent of the \type{\clef} command.
\startmylily
#(define* (instru->music instru #:optional (clef "treble"))
   (sim (make-clef-set clef)    ;; &\tfx\em{&sim = << … >>&}&
        global
        (obj->music instru)     ;; &\tfx\em{&notes&}&
        (add-dyn instru)))      &\tfx\em{&%% dynamics&}&
\stopmylily
Separate parts in treble clef can be edited simply with:
\startmylily \new Staff { $(instru->music 'vlI) } \stopmylily
The other parts will have to specify the key:
\startmylily
\new Staff { $(instru->music 'viola "alto") } ;; &\tfx\em{&alto clef&}&
\new Staff { $(instru->music 'vlc "bass") } ;; &\tfx\em{&bass clef&}&
\stopmylily
Note that if you have put \type{#(define part 'instru)} at the head of the file, as explained in the previous paragraph, you can replace the instrument name with the word \type{part}:
\startmylily \new Staff { $(instru->music part [clef]) } \stopmylily
\page %%%%%%%%%%%%%%%%%%%%%%%%%

%%%%%%%%%%%
\index{split-instru}
{\tfa\bf→} {\bf Main score : dealing with 2 instruments in a same staff}\\
The function below helps to avoid duplicate dynamics. It puts in one copy, the common dynamics at the bottom of the staff; only dynamics belonging only to the upper voice will be above the staff.
\startmylily
#(define* (split-instru instru1 instru2 #:optional (clef "treble"))
   (split                    ; &\tfx\em{& << … \\ … >>&}&
      (sim                   ; &\tfx\em{& << … >>&}&
         (make-clef-set clef)
         global
         dynamicUp           ; &\tfx\em{&dynamics direction UP&}&
         (add-dyn (list 'xor instru1 instru2))
         (obj->music instru1))
      (sim 
         (add-dyn instru2)
         (obj->music instru2))))
\new Staff { $(split-instru 'clarI 'clarII) }
\stopmylily
\blank[0.8em]
For a score with 3 horns for example, \type{instru->music} and \type{split-instru}~can be used:
\startmylily
\new StaffGroup <<
  \new Staff \with { instumentName = #"horn 1" } 
                 $(instru->music 'hornI)
  \new Staff \with { instumentName = 
                            \markup \vcenter {"horn " \column { 2 3 }}}
                 $(split-instru 'hornII 'hornIII) >>
\stopmylily

\index{part-combine-instru}
Instead of \type{split-instru}, a \type{part-combine-instru} function may be preferred.
\startmylily
#(define* (part-combine-instru instru1 instru2 #:optional (clef "treble"))
   (sim
     (make-clef-set clef)
     global
     (part-combine                ; &\tfx\em{&\partCombine&}&
       (sim                       ; &\tfx\em{&upper voice&}&
         ; partCombineApart       ; &\tfx\em{&working mode&}&
         partCombineAutomatic     ; &\tfx\em{&default mode&}&
         dynamicUp                ; &\tfx\em{&dynamics direction UP&}&
         (add-dyn (list 'xor instru1 instru2))
         (obj->music instru1))
       (obj->music instru2))      ; &\tfx\em{&lower voice&}&
     (add-dyn instru2)))
\stopmylily
A staff using this function will be easily tweak-able.
Supposing that this staff is shared by clarinets 2 and 3, you can add the following code in {\em SCORE.ly} (not in {\em NOTES.ily}):
\startmylily
#(begin ;; &\tfx\em{&partCombine settings for staff cl2-cl3 &}&
   (x-rm 'cl2 partCombineApart 60 '(82 3/8) 129)
   (x-rm 'cl2 partCombineChords 85)
   (x-rm 'cl2 partCombineAutomatic 61 86 138)
   …)
\stopmylily
\hairline
Warnings : \type{partCombineApart, partCombineChords, partCombineAutomatic…} are the new names in the most recent Lilypond versions. For Lilypond 2.20, you must use instead:\\ \type{partcombineApart, partcombineChords, partcombineAutomatic…}
\hairline
\page %%%%%%%%%%%%%%%%%%%
%%%%%%%
\starttitle[
  reference=addendum3,
  title={-ADDENDUM III-\\USING {\sca assocDynList}},
  list={ADDENDUM III : USING {\sca assocDynList}}
  ]

\reference[complément-assocDynList]{}
%{\tfa\bf→} {\bf Complément pour l'utilisation de \type{assocDynList}}
\blank[0.6em]
{\bf-} Adding customized dynamics :
\blank[0.3em]
\startmylily
pocodim = #(make-dynamic-script (markup #:normal-text #:italic "poco dim"))
piuf = #(make-dynamic-script (markup #:normal-text #:italic "più"
                                     #:dynamic "f"))
assocDynList = #(assoc-pos-dyn
  "1 f / 5 pocodim / 8 mf / (10 4) piuf / 12 fff" all)  % &\tfx\em{&(all instuments) &}& 
\stopmylily
\blank[0.7em]

{\bf-} Remove a dynamic and replace it with another:\\
In the above example, if we want to put \type{ff} bar 12 in the trumpet part instead of \type{fff}, we must first cancel the previous dynamic with an "empty" one, otherwise Lilypond will output an error:~2 dynamics at same place.
\blank[0.3em]
\startmylily
assocDynList = #(assoc-pos-dyn
  "1 f / 5 pocodim / 8 mf / 10 piuf / 12 fff" all ;&\tfx\em{& All (including the trumpet)&}& 
  "12 / 12 ff" 'tp )			               %&\tfx\em{& trumpet bar 12 : fff -> ff&}&  
\stopmylily
\blank[0.7em]

{\bf-} To reduce the number of dynamics in a score (for example when there is a large orchestral {\em crescendo}, containing "\type{cresc - - -}" in each instrument), one can use the \type{part?}  function described in addendum~II above, so that the suppression is only effective in the score and not in separate parts.
\blank[0.3em]
\startmylily
#(if (part? 'score) ;&\tfx\em{& the score is lightened bar 15 and 18&}&
   (set! assocDynList (append assocDynList (assoc-pos-dyn
     "15 / 18" '(&\tfx\em{& [list of instruments from which the dynamics are to be removed]&}& )))))
\stopmylily
\blank[0.7em]

{\bf-} Positions can be defined by variables (see \goto{\type{def-letters} function}[def-letters]\at{ page}[def-letters]) and they can be used in \type{assocDynList} without worrying about the characters {\tfb '} {\tfb `} or {\tfb ,} which are usually put in front of lists and symbols.
\startmylily 
A = #9           %&\tfx\em{& a bar number&}& 
B = #'(2 8 16)   %&\tfx\em{& a position inside a measure&}& 
assocDynList = #(assoc-pos-dyn
       "A p / (A 2 8) mp / (+ A 3) mf / ((+ A 3) 2 8) f" 'instruI
  ; => "9 p / (9 2 8) mp / 12 mf      / (12 2 8) f"
       "(cons 18 B) < / (cons 21 B) !" 'instruII
  ; => "(18 2 8 16) < / (21 2 8 16) !"
\stopmylily

\blank[1.2em]
\index{set-dyn}
\reference[set-dyn]

{\bf-} The addition of dynamics can be automated through the creation of a \type{set-dyn} function \footnote{Caution, despite its name, this function is {\em not} compatible \type{apply-to}}:
\startlily 
#(define ((set-dyn fmt0 . fmt-args) arg0 . args)
   (apply format (lst                ;&\tfx\em{& format arguments list&}&
     #f fmt0 fmt-args arg0 
     (filter not-procedure? args)))) %&\tfx\em{& syntax arg0 / arg1 / arg2… possible &}& 
\stoplily
The \type{fmt} parameter is a string that can contain escape sequences specific to the \type{format} scheme function. Thus, for example, each apparition of \type {~a} in \type{fmt}, will be successively replaced by the parameter \type{arg0, arg1, arg2 ...}, previously converted into a character string.\\ 
Here are a few possible uses.
\page %%%%%%%%%%%%%%%%%%%%%
%\blank
→ Copying the same dynamic in several places
\godown[0.4em]
\startmylily 
(map (set-dyn "(~a 4 8) f") '(13 28 42 55))
\stopmylily
\godown[0.4em]
This instruction returns a list of strings. So, to be able to include it as an argument of \type{assoc-pos-dyn}, it would be in theory necessary to group together each element of the resulting list into one string, with a slash \type{/} as separator. In practice, \type{assoc-pos-dyn} avoids this work, by performing this formatting itself when an argument is a list.

The following instruction:
\startmylily
assocDynList = #(assoc-pos-dyn
  (map (set-dyn "(~a 4 8) f") '(13 28 42 55)) instrus
  …)
\stopmylily
is therefore equivalent to:
\startmylily
assocDynList = #(assoc-pos-dyn
  "(13 4 8) f / (28 4 8) f / (42 4 8) f / (55 4 8) f" instrus
  …)
\stopmylily
However, the same result can be obtained by using only the escape sequences of the \type{format} function:
\startmylily 
((set-dyn "~@{~}" "(~a 4 8) f~^ / ") 13 28 42 55)
\stopmylily
The \type{~@{~}} sequence allows the following instruction to loop until the parameters are exhausted, and the \type{~^} sequence does not add the slash \type{/} when the last parameter is reached.\\
It is then easy to automate things by a function \type{x-dyn} for example:
\startmylily
#(define (x-dyn fmt) (set-dyn "~@{~}" (string-append fmt "~^ / ")))
\stopmylily
The use of this function is basic:
\startmylily
((x-dyn "(~a 4 8) f") 13 28 42 55)
\stopmylily
\blank
→ Copying a group of dynamics remaining within the same measure\\
An additional escape sequence \type{~:*} is used here to return to the previous parameter.
\blank[0.5em]
\startmylily
#(define fmt<> "(~a 8) < / (~:*~a 4 16) > / (~:*~a 2 16) !")
#(define dyn<> (set-dyn fmt<>))
assocDynList = #(assoc-pos-dyn
  (dyn<> 45) '(instru1 instru2)
  (map dyn<> '(47 49)) 'instru3
  …)
\stopmylily
which results in:
\startmylily
assocDynList = #(assoc-pos-dyn
  "(45 8) < / (45 4 16) > / (45 2 16) !" '(instru1 instru2)
  "(47 8) < / (47 4 16) > / (47 2 16) ! / 
   (49 8) < / (49 4 16) > / (49 2 16) !" 'instru3
  …)
\stopmylily
The same \quote{pattern} \type{fmt<>} can also be used with the previous \type{x-dyn} function. The result will be identical but the syntax will be slightly different
\blank[0.5em]
\startmylily
#(define dyn<> (x-dyn fmt<>))
assocDynList = #(assoc-pos-dyn
  (dyn<> 45) '(instru1 instru2)
  (dyn<> 47 49) 'instru3
  …)
\stopmylily
\blank
→ Copying a group of dynamics spanning several measures\\
\blank[0.4em]
A slash \type{/} has been added here to separate the arguments into groups of 3:
\blank[0.5em]
\startmylily
#(define dyn<> (x-dyn "~a < / ~a > / ~a !"))
assocDynList = #(assoc-pos-dyn
  (dyn<> 1 7 10 / '(11 4) '(13 8 16) '(17 8))) 'instru
  …)
\stopmylily
\page %%%%%%%%%%%%%%%%%%%%
The code results in:
\startmylily
assocDynList = #(assoc-pos-dyn
  "1 < / 7 > / 10 ! / (11 4) < / (13 8 16) > / (17 8) !" 'instru
  …)
\stopmylily
\index{list-offset}
The definition of \type{dyn<>} creating hairpins is here the most generic possible.\\
However, if in a piece, sequences of nuances are separated each time, by an identical number of bars (for example, a crescendo~\type{<} followed, 2 bars later, by a decrescendo~\type{>} ending at a 3\high{rd} bar), the use of the following function may be judicious:
\startmylily
#(define (bar-offset bar-numbers offsets)
   "(bar-offset '(b1 b2...bi) '(o1 o2 ...oj)) renvoie la liste :
     b1 + o1, b1 + o2,...b1 + oj, b2 + o1, b2 + o2,...b2 + oj ... bi + oj"
  (fold-right
    (lambda(b prev1)
      (fold-right
        (lambda(o prev2) (cons (+ b o) prev2))
        prev1
        offsets))
    '()
    bar-numbers))
\stopmylily

It will be possible then to define a \type{dyn<>} function with :
\startmylily
#(define (dyn<> . bar-nums )
   (apply (x-dyn "~a < / (~a 8) > / (~a 4 16) !")
          (bar-offset bar-nums '(0 2 3))))
\stopmylily
Inside a \type{assocDynList} code, this simple line : 
\startmylily "(map dyn<> '(5 11 20))" 'instru \stopmylily
will be equivalent to all this code:
\startmylily 
"5  < / (7  8) > / (8  4 16) ! /
 11 < / (13 8) > / (14 4 16) ! /
 20 < / (22 8) > / (23 4 16) !" 'instru
\stopmylily
\blank[1.8em]

\index{def-dyn}
{\bf-} On a large project, the definition of \type{assocDynList} can be quite large, and the definitions of our functions can be quite distant in the file from where they are used in \type{assocDynList}. It is nevertheless possible to define objects within the arguments of the  \type{assoc-pos-dyn} function, by the following macro \type{def-dyn}:
\startlily
#(define-macro (def-dyn dyn arg) `(begin (define ,dyn ,arg) /))
\stoplily
It can be used in the following way:
\startmylily
assocDynList = #(assoc-pos-dyn
  ...
  (def-dyn dyn1 (x-dyn "~a p < / ~a f"))
  (dyn1 5 6 / 9 10) 'instru1
  ...
  (def-dyn dyn2 (set-dyn "(7 2. 16) ~asf"))
  (dyn2 "^") 'instru1
  (dyn2 "_") 'instru2
  ...
)
\stopmylily
\blank[1.8em]
Other macros can be defined but like \type{def-dyn}, they must all have as return value, the function scheme \type{/} (division of numbers). \type{assoc-pos-dyn} indeed, automatically removes such an element from the list given as argument.\\
\blank
\index{def-dyn†txt}
\index{def-txt†dyn}
The following 2 macros \type{def-dyn+txt} and \type{def-txt+dyn} allow you to create a compound dynamic such as \type{p sub} or \type{molto f}.\\
They take 2 arguments of type {\em string}. The dynamic name is concatenated from these 2 arguments: \type{psub} and \type{moltof} for example.\\

\startlily
#(define-macro (def-dyn+txt dyn txt)   ;&\em{& (def-dyn+txt "p" "sub") => psub &}&
`(let ((sym (string->symbol (string-append ,dyn ,txt))))
   (ly:parser-define! sym (make-dynamic-script (markup
                            #:dynamic ,dyn
                            #:normal-text #:italic ,txt)))
   /))

#(define-macro (def-txt+dyn txt dyn) ;&\em{& (def-txt+dyn "molto" "f") => moltof&}&
`(let ((sym (string->symbol (string-append ,dyn ,txt))))
   (ly:parser-define! sym (make-dynamic-script (markup
                            #:normal-text #:italic ,txt
                            #:dynamic ,dyn)))
   /))
\stoplily
\blank[1.8em]

\index{def-span-dyn}
The macro \type{def-span-dyn} here allows to create dynamics with extender lines.\\
The 1\high{st} argument of type {\em symbol} is used to define the name of the new dynamic.\\
The 2\high{nd} argument of type {\em character string} is the text to be engraved by the dynamic.\\
If the 1\high{st} argument is omitted, the dynamic name is formed from the {\em string}, keeping only the letters in it.

\startlily
&\em{&% pre-function &}&
#(define (def-span-dyn-generic sym txt)
     (ly:parser-define! sym (make-music 'CrescendoEvent
       'span-direction START 'span-type 'text 'span-text txt))
     /)
&\em{&% the macro &}&
#(define-macro (def-span-dyn txt . args)
`(if (and (pair? (list ,@args)) (symbol? ,txt))
   (apply def-span-dyn-generic (list ,txt ,@args))
   (let ((sym (string->symbol (string-filter ,txt char-set:letter))))
     (def-span-dyn-generic sym ,txt))))

                           &\em{&%%%%%%%%%%%%%%%%%&}&
global = s1*3
music = { \repeat unfold 16 c8 c1 }

#(init '(instru))
#(rm 'instru 1 (rel 1 music))

assocDynList = #(assoc-pos-dyn
  (def-span-dyn 'crescspace " ")
  (def-span-dyn "cresc. molto")
    "1 pp crescspace / 2 crescmolto / 3 ff" 'instru
)

\new Staff $(sim global instru (add-dyn 'instru))
\stoplily
\midaligned{\externalfigure[macro-spanner-dyn.pdf][wfactor=300]}
%\midaligned{\underbar{\sc\Words exemple 11}}

%%%%%%%%%%%%%%%%%%%%%%%%%%%%%% 
\title[index]{INDEX}
\placeindex[level=title]
%%%%%%%%%%%%%%%%%%%%%%%%%%%%%%%

\stoptext

